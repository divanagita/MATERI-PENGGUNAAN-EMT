% Options for packages loaded elsewhere
\PassOptionsToPackage{unicode}{hyperref}
\PassOptionsToPackage{hyphens}{url}
\documentclass[
]{book}
\usepackage{xcolor}
\usepackage{amsmath,amssymb}
\setcounter{secnumdepth}{-\maxdimen} % remove section numbering
\usepackage{iftex}
\ifPDFTeX
  \usepackage[T1]{fontenc}
  \usepackage[utf8]{inputenc}
  \usepackage{textcomp} % provide euro and other symbols
\else % if luatex or xetex
  \usepackage{unicode-math} % this also loads fontspec
  \defaultfontfeatures{Scale=MatchLowercase}
  \defaultfontfeatures[\rmfamily]{Ligatures=TeX,Scale=1}
\fi
\usepackage{lmodern}
\ifPDFTeX\else
  % xetex/luatex font selection
\fi
% Use upquote if available, for straight quotes in verbatim environments
\IfFileExists{upquote.sty}{\usepackage{upquote}}{}
\IfFileExists{microtype.sty}{% use microtype if available
  \usepackage[]{microtype}
  \UseMicrotypeSet[protrusion]{basicmath} % disable protrusion for tt fonts
}{}
\makeatletter
\@ifundefined{KOMAClassName}{% if non-KOMA class
  \IfFileExists{parskip.sty}{%
    \usepackage{parskip}
  }{% else
    \setlength{\parindent}{0pt}
    \setlength{\parskip}{6pt plus 2pt minus 1pt}}
}{% if KOMA class
  \KOMAoptions{parskip=half}}
\makeatother
\usepackage{graphicx}
\makeatletter
\newsavebox\pandoc@box
\newcommand*\pandocbounded[1]{% scales image to fit in text height/width
  \sbox\pandoc@box{#1}%
  \Gscale@div\@tempa{\textheight}{\dimexpr\ht\pandoc@box+\dp\pandoc@box\relax}%
  \Gscale@div\@tempb{\linewidth}{\wd\pandoc@box}%
  \ifdim\@tempb\p@<\@tempa\p@\let\@tempa\@tempb\fi% select the smaller of both
  \ifdim\@tempa\p@<\p@\scalebox{\@tempa}{\usebox\pandoc@box}%
  \else\usebox{\pandoc@box}%
  \fi%
}
% Set default figure placement to htbp
\def\fps@figure{htbp}
\makeatother
\setlength{\emergencystretch}{3em} % prevent overfull lines
\providecommand{\tightlist}{%
  \setlength{\itemsep}{0pt}\setlength{\parskip}{0pt}}
\usepackage{bookmark}
\IfFileExists{xurl.sty}{\usepackage{xurl}}{} % add URL line breaks if available
\urlstyle{same}
\hypersetup{
  hidelinks,
  pdfcreator={LaTeX via pandoc}}

\author{}
\date{}

\begin{document}
\frontmatter

\mainmatter
\chapter{Kalkulus dengan EMT}\label{kalkulus-dengan-emt}

Materi Kalkulus mencakup di antaranya: - Fungsi (fungsi aljabar, trigonometri, eksponensial, logaritma, komposisi fungsi) - Limit Fungsi, - Turunan Fungsi, - Integral Tak Tentu, - Integral Tentu dan Aplikasinya, - Barisan dan Deret (kekonvergenan barisan dan deret).

EMT (bersama Maxima) dapat digunakan untuk melakukan semua perhitungan di dalam kalkulus, baik secara numerik maupun analitik (eksak).

\chapter{Fungsi Fungsi adalah pemetaan setiap anggota sebuah himpunan kepada}\label{fungsi-fungsi-adalah-pemetaan-setiap-anggota-sebuah-himpunan-kepada}

anggota himpunan yang lain. Fungsi adalah salah satu konsep dasar dari matematika dan setiap ilmu kuantitatif. Pada dasarnya, fungsi adalah suatu relasi yang memetakan setiap anggota dari suatu himpunan yang disebut sebagai daerah asal atau domain ke tepat satu anggota himpunan lain yang disebut daerah kawan atau kodomain.

Adapun beberapa jenis fungsi yaitu:

\section{Fungsi Aljabar}\label{fungsi-aljabar}

Fungsi aljabar adalah fungsi yang dapat didefinisikan sebagai akar dari sebuah persamaan aljabar. Fungsi aljabar merupakan ekspresi aljabar menggunakan sejumlah suku terbatas, yang melibatkan operasi aljabar seperti penambahan, pengurangan, perkalian, pembagian, dan peningkatan menjadi pangkat pecahan.

Sifat-Sifat Fungsi Aljabar

\begin{enumerate}
\def\labelenumi{\arabic{enumi}.}
\tightlist
\item
  Fungsi Injektif: Fungsi aljabar dapat menjadi injektif jika setiap elemen di domain dipetakan ke elemen yang berbeda di kodomain. Artinya, jika
\end{enumerate}

\[f(a1)=f(a2)\]

maka

\[a1=a2\]

\begin{enumerate}
\def\labelenumi{\arabic{enumi}.}
\setcounter{enumi}{1}
\tightlist
\item
  Fungsi Surjektif: Fungsi aljabar dapat menjadi surjektif jika setiap elemen di kodomain memiliki setidaknya satu elemen di domain yang memetakan ke sana. Artinya, untuk setiap
\end{enumerate}

b di kodomain, ada a di domain sehingga

\[f(a)=b\]

\begin{enumerate}
\def\labelenumi{\arabic{enumi}.}
\setcounter{enumi}{2}
\tightlist
\item
  Fungsi Bijektif: Fungsi aljabar yang juga bijektif harus memenuhi keduanya, yaitu injektif dan surjektif. Artinya, setiap elemen di domain dipetakan ke elemen unik di kodomain, dan setiap elemen di kodomain dipetakan oleh elemen di domain
\end{enumerate}

Fungsi aljabar dapat memiliki berbagai bentuk kurva tergantung pada jenisnya. Berikut beberapa contoh:

\begin{enumerate}
\def\labelenumi{\arabic{enumi}.}
\tightlist
\item
  Fungsi Polinomial
\end{enumerate}

\[y=4x^3-8x^2-3x+2\]\textgreater plot2d(``4*x\textsuperscript{3-8*x}2-3*x+2''):

\begin{figure}
\centering
\pandocbounded{\includegraphics[keepaspectratio]{images/KALKULUS_Diva Nagita(23030630024)-005.png}}
\caption{images/KALKULUS\_Diva\%20Nagita(23030630024)-005.png}
\end{figure}

\begin{enumerate}
\def\labelenumi{\arabic{enumi}.}
\setcounter{enumi}{1}
\tightlist
\item
  Fungsi Rasional
\end{enumerate}

\[f(x)=\frac{x^2+1}{x-1}\]\textgreater reset;

\textgreater plot2d(``(x\^{}2+1)/(x-1)''):

\begin{figure}
\centering
\pandocbounded{\includegraphics[keepaspectratio]{images/KALKULUS_Diva Nagita(23030630024)-007.png}}
\caption{images/KALKULUS\_Diva\%20Nagita(23030630024)-007.png}
\end{figure}

\begin{enumerate}
\def\labelenumi{\arabic{enumi}.}
\setcounter{enumi}{2}
\tightlist
\item
  Fungsi Radikal
\end{enumerate}

\[\sqrt{x}\]\textgreater reset;

\textgreater plot2d(``sqrt(x)''):

\begin{figure}
\centering
\pandocbounded{\includegraphics[keepaspectratio]{images/KALKULUS_Diva Nagita(23030630024)-009.png}}
\caption{images/KALKULUS\_Diva\%20Nagita(23030630024)-009.png}
\end{figure}

\section{Fungsi Trigonometri}\label{fungsi-trigonometri}

Fungsi trigonometri adalah fungsi matematika yang mengaitkan sudut dari segitiga siku-siku dengan perbandingan antara dua sisi segitiga, serta dapat didefinisikan melalui lingkaran satuan. Fungsi ini mencakup enam jenis utama: sinus (sin), kosinus (cos), tangen (tan), kosekan (csc), sekan (sec), dan kotangen (cot).

Berikut adalah beberapa sifat utama fungsi trigonometri,

\begin{itemize}
\item
  Periodisitas: Fungsi sinus dan kosinus memiliki periode 2pi, sedangkan tangen memiliki periode pi.
\item
  Amplitudo: Jarak dari garis tengah ke titik maksimum atau minimum; untuk sinus dan kosinus, amplitudonya adalah 1.
\item
  Nilai Maksimum dan Minimum: Sinus dan kosinus memiliki nilai maksimum +1 dan minimum -1; tangen tidak terbatas.
\end{itemize}

Titik Asimtot: Terdapat pada fungsi tangen di

\[\frac{\pi}{2} +k{\pi}\]

untuk setiap bilangan bulat k.

Fungsi trigonometri memiliki berbagai bentuk kurva tergantung pada jenisnya. Berikut beberapa contoh:

\begin{enumerate}
\def\labelenumi{\arabic{enumi}.}
\tightlist
\item
  Sinus
\end{enumerate}

\textgreater sin(90°)

\begin{verbatim}
1
\end{verbatim}

\textgreater plot2d(``sin(x*pi/180)'', 0, 90):

\begin{figure}
\centering
\pandocbounded{\includegraphics[keepaspectratio]{images/KALKULUS_Diva Nagita(23030630024)-011.png}}
\caption{images/KALKULUS\_Diva\%20Nagita(23030630024)-011.png}
\end{figure}

\begin{enumerate}
\def\labelenumi{\arabic{enumi}.}
\setcounter{enumi}{1}
\tightlist
\item
  Kosinus
\end{enumerate}

\textgreater cos(60)

\begin{verbatim}
-0.952412980415
\end{verbatim}

\textgreater plot2d(``cos(x*pi/180)'', 0, 60):

\begin{figure}
\centering
\pandocbounded{\includegraphics[keepaspectratio]{images/KALKULUS_Diva Nagita(23030630024)-012.png}}
\caption{images/KALKULUS\_Diva\%20Nagita(23030630024)-012.png}
\end{figure}

\begin{enumerate}
\def\labelenumi{\arabic{enumi}.}
\setcounter{enumi}{2}
\tightlist
\item
  Tangen
\end{enumerate}

\textgreater tan(45°)

\begin{verbatim}
1
\end{verbatim}

\textgreater plot2d(``tan(x*pi/180)'', 0, 45):

\begin{figure}
\centering
\pandocbounded{\includegraphics[keepaspectratio]{images/KALKULUS_Diva Nagita(23030630024)-013.png}}
\caption{images/KALKULUS\_Diva\%20Nagita(23030630024)-013.png}
\end{figure}

\section{Fungsi Eksponensial}\label{fungsi-eksponensial}

Fungsi eksponensial adalah fungsi matematika yang memiliki bentuk umum:

\[f(x)=a^x\]

dimana a adalah bilangan positif yang disebut basis (biasanya), dan x adalah eksponen.

Adapun sifat-sifat dari fungsi eksponensial adalah

\begin{itemize}
\item
  Untuk semua nilai x, fungsi eksponensial selalu menghasilkan nilai positif, yaitu f(x)\textgreater0 untuk setiap x.
\item
  Jika a\textgreater1, fungsi eksponensial akan meningkat.
\item
  Jika 0\textless a\textless1, fungsi eksponensial akan menurun.
\item
  Setiap bilangan yang dipangkatkan nol sama dengan satu, yaitu
\end{itemize}

\[a^0=1\]* Setiap bilangan yang dipangkatkan satu sama dengan bilangan itu * sendiri, yaitu * a\^{}1=a

\[a^1=a\]* Untuk basis yang sama, berlaku sifat penjumlahan: * a\^{}m . a\textsuperscript{n=a}\{m+n\}

\[a^m . a^n=a^{m+n}\]* Untuk pembagian dengan basis yang sama, berlaku sifat pengurangan: * \frac{a^m}{a^n}=a\^{}\{m-n\}

\[\frac{a^m}{a^n}=a^{m-n}\]* Grafik fungsi eksponensial mendekati sumbu x (y = 0) tetapi tidak * pernah menyentuhnya, sehingga memiliki asimtot horizontal di y = 0.

\begin{itemize}
\tightlist
\item
  Turunan dari fungsi eksponensial dengan basis e adalah unik karena
\item
  hasilnya adalah fungsi itu sendiri, yaitu
\item
  \frac{d}{dx}e\^{}x = e\^{}x
\end{itemize}

\[\frac{d}{dx}e^x = e^x\]Contoh Kurva fungsi eksponensial:

\[4^x\]\textgreater plot2d(``4\^{}x'', -2, 2):

\begin{figure}
\centering
\pandocbounded{\includegraphics[keepaspectratio]{images/KALKULUS_Diva Nagita(23030630024)-021.png}}
\caption{images/KALKULUS\_Diva\%20Nagita(23030630024)-021.png}
\end{figure}

\section{Fungsi Logaritma}\label{fungsi-logaritma}

Fungsi logaritma adalah fungsi matematis yang merupakan invers dari fungsi eksponensial. Secara formal, jika kita memiliki fungsi eksponensial

\[f(x)=a^x\]

(dengan a\textgreater0 dan atidak sama dengan 1), maka fungsi logaritma dapat dinyatakan sebagai

\[f^{-1}(x)=log_a(x)\]

yang berarti logaritma dari x dengan basis a adalah eksponen yang harus dipangkatkan pada a untuk mendapatkan x. Fungsi logaritma memiliki bentuk umum

\[f(x)=a log_b (x)\]

dengan syarat x\textgreater0.

Fungsi logaritma memiliki beberapa sifat penting:

\begin{enumerate}
\def\labelenumi{\arabic{enumi}.}
\tightlist
\item
  Sifat Penjumlahan:
\end{enumerate}

\[log_a (m.n)=log_a(m)+log_a(n)\]

\begin{enumerate}
\def\labelenumi{\arabic{enumi}.}
\setcounter{enumi}{1}
\tightlist
\item
  Sifat Pengurangan:
\end{enumerate}

\[log_a (\frac{m}{n})=log_a(m)-log_a(n)\]

\begin{enumerate}
\def\labelenumi{\arabic{enumi}.}
\setcounter{enumi}{2}
\tightlist
\item
  Sifat Pangkat:
\end{enumerate}

\[log_a(m^p)=p.log_a(m)\]

\begin{enumerate}
\def\labelenumi{\arabic{enumi}.}
\setcounter{enumi}{3}
\tightlist
\item
  Perubahan Basis:
\end{enumerate}

\[log_a(b)=\frac {log_c(b)}{log_c(a)}\]

untuk basis yang berbeda.

Berikut contoh kurva fungsi logaritma sederhana:

\textgreater plot2d(``log10(x)'', 0.01, 100):

\begin{figure}
\centering
\pandocbounded{\includegraphics[keepaspectratio]{images/KALKULUS_Diva Nagita(23030630024)-029.png}}
\caption{images/KALKULUS\_Diva\%20Nagita(23030630024)-029.png}
\end{figure}

\textgreater plot2d(``ln(x)'', ``log10(x)'', 1, 10):

\begin{figure}
\centering
\pandocbounded{\includegraphics[keepaspectratio]{images/KALKULUS_Diva Nagita(23030630024)-030.png}}
\caption{images/KALKULUS\_Diva\%20Nagita(23030630024)-030.png}
\end{figure}

\section{Komposisi Fungsi}\label{komposisi-fungsi}

Komposisi fungsi adalah operasi yang menggabungkan dua atau lebih fungsi untuk membentuk fungsi baru. Dalam notasi, jika terdapat dua fungsi f dan g, komposisi fungsi ditulis sebagai (f o g)(x),

yang berarti kita pertama-tama menerapkan fungsi x, kemudian hasilnya digunakan sebagai input untuk fungsi f.~Secara matematis, ini dapat dinyatakan sebagai (f o g)(x)=f(g(x)).

Adapun sifat-sifat komposisi fungsi

\begin{enumerate}
\def\labelenumi{\arabic{enumi}.}
\tightlist
\item
  Asosiatif:
\end{enumerate}

\[(f o(g o h))(x)=((f o g)o h)(x)\]

untuk semua fungsi f,g,h yang terdefinisi dengan baik.

\begin{enumerate}
\def\labelenumi{\arabic{enumi}.}
\setcounter{enumi}{1}
\tightlist
\item
  Identitas:
\end{enumerate}

\[(f o I)(x)=I(x)\]

dimana I(x) adalah fungsi identitas yang memetakan setiap elemen ke dirinya sendiri.

\begin{enumerate}
\def\labelenumi{\arabic{enumi}.}
\setcounter{enumi}{2}
\tightlist
\item
  Tidak komutatif:
\end{enumerate}

\[(f o g)(x)\]

tidak sama dengan

\[(g o h)(x)\]

kecuali dalam kasus tertentu di mana kedua fungsi tersebut saling berkomutasi.

\begin{enumerate}
\def\labelenumi{\arabic{enumi}.}
\setcounter{enumi}{3}
\tightlist
\item
  Bijektif: Jika dua fungsi yang dikomposisikan adalah bijektif, maka hasil komposisinya juga bijektif. Ini berarti bahwa jika
\end{enumerate}

f dan g masing-masing injektif dan surjektif, maka f o g juga memiliki sifat ini.

Berikut contoh kurva komposisi fungsi:

\begin{enumerate}
\def\labelenumi{\arabic{enumi}.}
\tightlist
\item
\end{enumerate}

\[f(x)=x^2\]\[g(x)=sin x\]\[h(x)=f(g(x))\]\textgreater function f(x) := x\^{}2

\textgreater function g(x) := sin(x)

\textgreater function h(x) := f(g(x))

\textgreater plot2d(``h(x)'', -pi, pi):

\begin{figure}
\centering
\pandocbounded{\includegraphics[keepaspectratio]{images/KALKULUS_Diva Nagita(23030630024)-038.png}}
\caption{images/KALKULUS\_Diva\%20Nagita(23030630024)-038.png}
\end{figure}

\begin{enumerate}
\def\labelenumi{\arabic{enumi}.}
\setcounter{enumi}{1}
\tightlist
\item
\end{enumerate}

\[f(x)=cos x\]\[g(x)=x^3-1\]\[f(g(x))\]\textgreater function f(x) := cos(x)

\textgreater function g(x) := x\^{}3 - 1

\textgreater function h(x) := f(g(x))

\textgreater plot2d(``h(x)'', -1, 2):

\begin{figure}
\centering
\pandocbounded{\includegraphics[keepaspectratio]{images/KALKULUS_Diva Nagita(23030630024)-042.png}}
\caption{images/KALKULUS\_Diva\%20Nagita(23030630024)-042.png}
\end{figure}

\section{Limit Fungsi}\label{limit-fungsi}

Limit adalah konsep dasar dalam kalkulus yang menggambarkan perilaku suatu fungsi saat mendekati nilai tertentu. Limit digunakan untuk memahami perubahan nilai fungsi ketika variabel mendekati suatu titik. Limit dari suatu fungsi f(x)pada titik c adalah nilai yang didekati oleh f(x) saat mendekati c.~Notasi limit dinyatakan sebagai:

\[\lim_{x\to c}f(x)=L\]1. Limit Kiri dan Kanan

\begin{itemize}
\tightlist
\item
  Limit kiri: limit dari fungsi(x) saat x mendekati c dari sisi kiri atau nilai x lebih kecil dari c dilambangkan dengan
\end{itemize}

\[\lim_{x\to c^-}f(x)=L^-\]* Limit kanan: limit dari fungsi(x) saat x mendekati c dari sisi kanan * atau nilai x lebih besar dari c dilambangkan dengan * \lim\_\{x\to c\textsuperscript{+\}f(x)=L}+

\[\lim_{x\to c^+}f(x)=L^+\]Perhitungan limit pada EMT dapat dilakukan dengan menggunakan fungsi Maxima, yakni ``limit''. Fungsi ``limit'' dapat digunakan untuk menghitung limit fungsi dalam bentuk ekspresi maupun fungsi yang sudah didefinisikan sebelumnya. Nilai limit dapat dihitung pada sebarang nilai atau pada tak hingga (-inf, minf, dan inf). Limit kiri dan limit kanan juga dapat dihitung, dengan cara memberi opsi ``plus'' atau ``minus''. Hasil limit dapat berupa nilai, ``und'' (tak definisi), ``ind'' (tak tentu namun terbatas), ``infinity'' (kompleks tak hingga).

Berikut visualisasi limit ke :

\begin{enumerate}
\def\labelenumi{\arabic{enumi}.}
\tightlist
\item
  Fungsi Aljabar
\end{enumerate}

\[\lim_{x \to 1} \frac{x^2 - 1}{x - 1} = 2\]2. Fungsi Trigonometri

\[\lim_{x \to 0} \frac{\sin(x)}{x} = 1\]3. Fungsi Eksponensial

\[\lim_{x \to -\infty} 2^x = 0\]4. Fungsi Logaritma

\[\lim_{x \to 0^+} \log(x) = -\infty\]5. Komposisi Fungsi

\[\lim_{x \to \frac{\pi}{2}} (g(x))^2 = 1\]\textgreater\$limit((x\^{}2-1)/(x-1),x,1)

\[2\]\textgreater function f(x):= x\^{}2-1/x-1

\textgreater aspect(1.0); plot2d(``(x\^{}2-1)/(x-1)'',0,1); plot2d(1,2,\textgreater points,style=``ow'',\textgreater add):

\begin{figure}
\centering
\pandocbounded{\includegraphics[keepaspectratio]{images/KALKULUS_Diva Nagita(23030630024)-052.png}}
\caption{images/KALKULUS\_Diva\%20Nagita(23030630024)-052.png}
\end{figure}

\textgreater\$limit((sin (x))/(x),x,0)

\[1\]\textgreater function f(x):= sin(x)/x

\textgreater aspect(1.5); plot2d(``(sin(x))/(x)'',0,1); plot2d(0,1,\textgreater points,style=``ow'',\textgreater add):

\begin{figure}
\centering
\pandocbounded{\includegraphics[keepaspectratio]{images/KALKULUS_Diva Nagita(23030630024)-054.png}}
\caption{images/KALKULUS\_Diva\%20Nagita(23030630024)-054.png}
\end{figure}

\textgreater\$showev('limit(sqrt(x\^{}2-3*x)/(x+1),x,inf))

\[\lim_{x\rightarrow \infty }{\frac{\sqrt{-3\,x+x^2}}{1+x}}=1\]\textgreater\$limit((x\textsuperscript{3-13*x}2+51*x-63)/(x\textsuperscript{3-4*x}2-3*x+18),x,3)

\[-\frac{4}{5}\]\textgreater aspect(1.5); plot2d(``(x\textsuperscript{3-13*x}2+51*x-63)/(x\textsuperscript{3-4*x}2-3*x+18)'',0,4); plot2d(3,-4/5,\textgreater points,style=``ow'',\textgreater add):

\begin{figure}
\centering
\pandocbounded{\includegraphics[keepaspectratio]{images/KALKULUS_Diva Nagita(23030630024)-057.png}}
\caption{images/KALKULUS\_Diva\%20Nagita(23030630024)-057.png}
\end{figure}

\textgreater\$limit(2*x*sin(x)/(1-cos(x)),x,0)

\[4\]\textgreater plot2d(``2*x*sin(x)/(1-cos(x))'',-pi,pi); plot2d(0,4,\textgreater points,style=``ow'',\textgreater add):

\begin{figure}
\centering
\pandocbounded{\includegraphics[keepaspectratio]{images/KALKULUS_Diva Nagita(23030630024)-059.png}}
\caption{images/KALKULUS\_Diva\%20Nagita(23030630024)-059.png}
\end{figure}

\textgreater\$limit(cot(7*h)/cot(5*h),h,0)

\[\frac{5}{7}\]\textgreater plot2d(``cot(7*x)/cot(5*x)'',-0.001,0.001); plot2d(0,5/7,\textgreater points,style=``ow'',\textgreater add):

\begin{figure}
\centering
\pandocbounded{\includegraphics[keepaspectratio]{images/KALKULUS_Diva Nagita(23030630024)-061.png}}
\caption{images/KALKULUS\_Diva\%20Nagita(23030630024)-061.png}
\end{figure}

\textgreater\$showev('limit(((x/8)\^{}(1/3)-1)/(x-8),x,8))

\[\lim_{x\rightarrow 8}{\frac{-1+\frac{x^{\frac{1}{3}}}{2}}{-8+x}}=  \frac{1}{24}\]\textgreater\$showev('limit(1/(2*x-1),x,0))

\[\lim_{x\rightarrow 0}{\frac{1}{-1+2\,x}}=-1\]\textgreater\$showev('limit((x\^{}2-3*x-10)/(x-5),x,5))

\[\lim_{x\rightarrow 5}{\frac{-10-3\,x+x^2}{-5+x}}=7\]\textgreater\$showev('limit(sqrt(x\^{}2+x)-x,x,inf))

\[\lim_{x\rightarrow \infty }{-x+\sqrt{x+x^2}}=\frac{1}{2}\]\textgreater\$showev('limit(abs(x-1)/(x-1),x,1,minus))

\[\lim_{x\uparrow 1}{\frac{\left| -1+x\right| }{-1+x}}=-1\]\textgreater\$showev('limit(sin(x)/x,x,0))

\[\lim_{x\rightarrow 0}{\frac{\sin x}{x}}=1\]\textgreater plot2d(``sin(x)/x'',-pi,pi); plot2d(0,1,\textgreater points,style=``ow'',\textgreater add):

\begin{figure}
\centering
\pandocbounded{\includegraphics[keepaspectratio]{images/KALKULUS_Diva Nagita(23030630024)-068.png}}
\caption{images/KALKULUS\_Diva\%20Nagita(23030630024)-068.png}
\end{figure}

\textgreater\$showev('limit(sin(x\^{}3)/x,x,0))

\[\lim_{x\rightarrow 0}{\frac{\sin x^3}{x}}=0\]\textgreater\$showev('limit(log(x), x, minf))

\[\lim_{x\rightarrow  -\infty }{\log x}={\it infinity}\]\textgreater\$showev('limit((-2)\^{}x,x, inf))

\[\lim_{x\rightarrow \infty }{\left(-2\right)^{x}}={\it infinity}\]\textgreater\$showev('limit(t-sqrt(2-t),t,2,minus))

\[\lim_{t\uparrow 2}{-\sqrt{2-t}+t}=2\]\textgreater\$showev('limit(t-sqrt(2-t),t,2,plus))

\[\lim_{t\downarrow 2}{-\sqrt{2-t}+t}=2\]\textgreater\$showev('limit(t-sqrt(2-t),t,5,plus)) // Perhatikan hasilnya

\[\lim_{t\downarrow 5}{-\sqrt{2-t}+t}=5-\sqrt{3}\,i\]\textgreater plot2d(``x-sqrt(2-x)'',0,2):

\begin{figure}
\centering
\pandocbounded{\includegraphics[keepaspectratio]{images/KALKULUS_Diva Nagita(23030630024)-075.png}}
\caption{images/KALKULUS\_Diva\%20Nagita(23030630024)-075.png}
\end{figure}

\textgreater\$showev('limit((x\textsuperscript{2-9)/(2*x}2-5*x-3),x,3))

\[\lim_{x\rightarrow 3}{\frac{-9+x^2}{-3-5\,x+2\,x^2}}=\frac{6}{7}\]\textgreater\$showev('limit((1-cos(x))/x,x,0))

\[\lim_{x\rightarrow 0}{\frac{1-\cos x}{x}}=0\]\textgreater\$showev('limit((x\textsuperscript{2+abs(x))/(x}2-abs(x)),x,0))

\[\lim_{x\rightarrow 0}{\frac{x^2+\left| x\right| }{x^2-\left| x  \right| }}=-1\]\textgreater\$showev('limit((1+1/x)\^{}x,x,inf))

\[\lim_{x\rightarrow \infty }{\left(1+\frac{1}{x}\right)^{x}}=e\]\textgreater plot2d(``(1+1/x)\^{}x'',0,1000):

\begin{figure}
\centering
\pandocbounded{\includegraphics[keepaspectratio]{images/KALKULUS_Diva Nagita(23030630024)-080.png}}
\caption{images/KALKULUS\_Diva\%20Nagita(23030630024)-080.png}
\end{figure}

\textgreater\$showev('limit((1+k/x)\^{}x,x,inf))

\[\lim_{x\rightarrow \infty }{\left(1+\frac{1}{r\,x}\right)^{x}}=e^{  \frac{1}{r}}\]\textgreater\$showev('limit((1+x)\^{}(1/x),x,0))

\[\lim_{x\rightarrow 0}{\left(1+x\right)^{\frac{1}{x}}}=e\]\textgreater\$showev('limit((x/(x+k))\^{}x,x,inf))

\[\lim_{x\rightarrow \infty }{\left(\frac{x}{\frac{1}{r}+x}\right)^{x  }}=e^ {- \frac{1}{r} }\]\textgreater\$showev('limit((E\textsuperscript{x-E}2)/(x-2),x,2))

\[\lim_{x\rightarrow 2}{\frac{-e^2+e^{x}}{-2+x}}=e^2\]\textgreater\$showev('limit(sin(1/x),x,0))

\[\lim_{x\rightarrow 0}{\sin \left(\frac{1}{x}\right)}={\it ind}\]\textgreater\$showev('limit(sin(1/x),x,inf))

\[\lim_{x\rightarrow \infty }{\sin \left(\frac{1}{x}\right)}=0\]\textgreater plot2d(``sin(1/x)'',-0.001,0.001):

\begin{figure}
\centering
\pandocbounded{\includegraphics[keepaspectratio]{images/KALKULUS_Diva Nagita(23030630024)-087.png}}
\caption{images/KALKULUS\_Diva\%20Nagita(23030630024)-087.png}
\end{figure}

\section{Limit Kanan}\label{limit-kanan}

Seperti yang sudah diketahui bahwa bentuk limit kanan adalah

\[\lim_{x\to c^+}f(x)=L^+\]

merupakan nilai yang didekati oleh f(x) saat x mendekati a dari arah kanan (atau dari yang lebih besar dari a).

Contohnya:

\[\lim_{x\to 2^+} \frac {1}{x}\]\textgreater reset;

\textgreater\$showev('limit(x-sqrt(2-x),x,2,plus))

\[\lim_{x\downarrow 2}{-\sqrt{2-x}+x}=2\]\textgreater plot2d(``x-sqrt(2-x)'',-pi,pi); plot2d(0,2,\textgreater points,style=``ow'',\textgreater add):

\begin{figure}
\centering
\pandocbounded{\includegraphics[keepaspectratio]{images/KALKULUS_Diva Nagita(23030630024)-091.png}}
\caption{images/KALKULUS\_Diva\%20Nagita(23030630024)-091.png}
\end{figure}

\section{Limit Kiri}\label{limit-kiri}

Seperti yang sudah diketahui bahwa bentuk limit kanan adalah

\[\lim_{x\to c^-}f(x)=L^-\]

merupakan nilai yang didekati oleh f(x) saat x mendekati a dari arah kiri (atau dari yang lebih kecil dari a).

Contoh:

\[\lim_{x\to 1^-}2^x\]\textgreater reset;

\textgreater\$showev('limit(abs(2\^{}x),x,2,minus))

\[\lim_{x\uparrow 2}{2^{x}}=4\]\textgreater plot2d(``2\^{}x'',-pi,pi); plot2d(0,2,\textgreater points,style=``ow'',\textgreater add):

\begin{figure}
\centering
\pandocbounded{\includegraphics[keepaspectratio]{images/KALKULUS_Diva Nagita(23030630024)-095.png}}
\caption{images/KALKULUS\_Diva\%20Nagita(23030630024)-095.png}
\end{figure}

\chapter{Turunan Fungsi Turunan merupakan salah satu materi lanjutan dari}\label{turunan-fungsi-turunan-merupakan-salah-satu-materi-lanjutan-dari}

limit fungsi.

Turunan dapat disebut juga sebagai diferensial dan proses dalam menentukan turunan suatu fungsi disebut sebagai diferensiasi.

Definisi turunan (limit):

\[f'(x) = \lim_{h\to 0} \frac{f(x+h)-f(x)}{h}\]Berikut adalah beberapa sifat turunan fungsi yang dapat diperhatikan sehingga memudahkan dalam melakukan operasi turunan fungsi antara lain:

\begin{enumerate}
\def\labelenumi{\arabic{enumi}.}
\tightlist
\item
  Aturan Penjumlahan
\end{enumerate}

\[\ (f+g)'(x)=f'(x)+g'(x)\\]

\begin{enumerate}
\def\labelenumi{\arabic{enumi}.}
\setcounter{enumi}{1}
\tightlist
\item
  Aturan Pengurangan
\end{enumerate}

\[\ (f-g)'(x)=f'(x)-g'(x)\\]

\begin{enumerate}
\def\labelenumi{\arabic{enumi}.}
\setcounter{enumi}{2}
\tightlist
\item
  Aturan Perkalian
\end{enumerate}

\[\ (f \cdot g)'(x)=f'(x) \cdot g(x)+f(x) \cdot g'(x)\\]

\begin{enumerate}
\def\labelenumi{\arabic{enumi}.}
\setcounter{enumi}{3}
\tightlist
\item
  Aturan Pembagian
\end{enumerate}

\[\ (\frac{f}{g})'(x)= \frac{f'(x) \cdot g(x)+f(x) \cdot g'(x)}{(g(x))^2}\] (dengan syarat g(x) tidak sama dengan 0)

\begin{enumerate}
\def\labelenumi{\arabic{enumi}.}
\setcounter{enumi}{4}
\item
  Aturan Rantai

  Jika g(x) adalah fungsi dalam fungsi f(g(x))
\end{enumerate}

\[\ (f(g(x)))'=f'(g(x)) \cdot g'(x)\\]

\begin{enumerate}
\def\labelenumi{\arabic{enumi}.}
\setcounter{enumi}{5}
\tightlist
\item
  Turunan dari konstanta
\end{enumerate}

\[\ \frac{d}{dx}(c)=0, c=konstanta\\]1. Aturan Penjumlahan

\begin{verbatim}
Contoh:


Terdapat f(x)=2x^2 dan g(x)=3x^2 tentukan nilai (f+g)'(x)
\end{verbatim}

\textgreater\$showev('limit((2*(x+h)\textsuperscript{2-2*x}2)/h,h,0)+'limit((3*(x+h)\textsuperscript{2-3*x}2)/h,h,0))

\[\lim_{h\rightarrow 0}{\frac{-2\,x^2+2\,\left(h+x\right)^2}{h}}+  \lim_{h\rightarrow 0}{\frac{-3\,x^2+3\,\left(h+x\right)^2}{h}}=10\,x\]\textgreater function f(x) \&= 2*x\^{}2

\begin{verbatim}
                                    2
                                 2 x
\end{verbatim}

\textgreater function g(x) \&= 3*x\^{}2

\begin{verbatim}
                                    2
                                 3 x
\end{verbatim}

\textgreater function h(x) \&= 10*x

\begin{verbatim}
                                 10 x
\end{verbatim}

\textgreater plot2d({[}``f(x)'',``g(x)'',``h(x)''{]}):

\begin{figure}
\centering
\pandocbounded{\includegraphics[keepaspectratio]{images/KALKULUS_Diva Nagita(23030630024)-104.png}}
\caption{images/KALKULUS\_Diva\%20Nagita(23030630024)-104.png}
\end{figure}

\begin{enumerate}
\def\labelenumi{\arabic{enumi}.}
\setcounter{enumi}{1}
\item
  Aturan Pengurangan

  Contoh:

  Terdapat f(x)=3x\^{}2+3 dan g(x)=2x\^{}2+2 tentukan nilai dari (f-g)'(x)
\end{enumerate}

\textgreater\$showev('limit((3*(x+h)\textsuperscript{2+3-(3*x}2+3))/h,h,0)-'limit((2*(x+h)\textsuperscript{2+2-(2*x}2+2))/h,h,0))

\[-\lim_{h\rightarrow 0}{\frac{-2\,x^2+2\,\left(h+x\right)^2}{h}}+  \lim_{h\rightarrow 0}{\frac{-3\,x^2+3\,\left(h+x\right)^2}{h}}=2\,x\]\textgreater function f(x)\&= 3*x\^{}2+3

\begin{verbatim}
                                      2
                               3 + 3 x
\end{verbatim}

\textgreater function g(x) \&= 2*x\^{}2+2

\begin{verbatim}
                                      2
                               2 + 2 x
\end{verbatim}

\textgreater function h(x) \&= 2*x

\begin{verbatim}
                                 2 x
\end{verbatim}

\textgreater plot2d({[}``f(x)'',``g(x)'',``h(x)''{]}):

\begin{figure}
\centering
\pandocbounded{\includegraphics[keepaspectratio]{images/KALKULUS_Diva Nagita(23030630024)-106.png}}
\caption{images/KALKULUS\_Diva\%20Nagita(23030630024)-106.png}
\end{figure}

\begin{enumerate}
\def\labelenumi{\arabic{enumi}.}
\setcounter{enumi}{2}
\item
  Aturan Perkalian

  Contoh:

  Terdapat u(x)=(x\textsuperscript{3-2).(x}2-3) tentukan nilai u'(x)

  Kita bisa menentukan terlebih dahulu f(x) dan g(x) dimana

  f(x)=x\^{}3-2 dan g(x)=x\^{}2-3 kemudian bisa kita terapkan aturan

  perkalian
\end{enumerate}

\textgreater\$showev((('limit(((x+h)\textsuperscript{3-2-(x}3-2))/h,h,0))*(x\textsuperscript{2-3))+((x}3-2)*'limit(((x+h)\textsuperscript{2-3-(x}2-3))/h,h,0)))

\[\left(-2+x^3\right)\,\left(\lim_{h\rightarrow 0}{\frac{-x^2+\left(h  +x\right)^2}{h}}\right)+\left(-3+x^2\right)\,\left(\lim_{h  \rightarrow 0}{\frac{-x^3+\left(h+x\right)^3}{h}}\right)=3\,x^2\,  \left(-3+x^2\right)+2\,x\,\left(-2+x^3\right)\]\textgreater\$expand(((2*x)*(x\textsuperscript{3-2))+((3*x}2)*(x\^{}2-3)))

\[-4\,x-9\,x^2+5\,x^4\]\textgreater function u(x) \&= (x\textsuperscript{3-2)*(x}2-3)

\begin{verbatim}
                                2          3
                        (- 3 + x ) (- 2 + x )
\end{verbatim}

\textgreater function h(x) \&= 2*x*(x\textsuperscript{3-2)+3*x}2*(x\^{}2-3)

\begin{verbatim}
                      2         2                3
                   3 x  (- 3 + x ) + 2 x (- 2 + x )
\end{verbatim}

\textgreater plot2d({[}``u(x)'',``h(x)''{]}):

\begin{figure}
\centering
\pandocbounded{\includegraphics[keepaspectratio]{images/KALKULUS_Diva Nagita(23030630024)-109.png}}
\caption{images/KALKULUS\_Diva\%20Nagita(23030630024)-109.png}
\end{figure}

\begin{enumerate}
\def\labelenumi{\arabic{enumi}.}
\setcounter{enumi}{3}
\item
  Aturan Pembagian

  Contoh:

  Terdapat u(x)=2x\^{}3/x-1 tentukan nilai u'(x)

  Kita bisa menentukan terlebih dahulu f(x) dan g(x) dimana

  f(x)=2x\^{}3 dan g(x)=x-1 kemudian bisa kita terapkan aturan pembagian
\end{enumerate}

\textgreater\$showev(((('limit((2*(x+h)\textsuperscript{3-2*x}3)/h,h,0))*(x-1))-((2*x\textsuperscript{3)*'limit(((x+h)-1-(x-1))/h,h,0)))/(x-1)}2)

\[\frac{-2\,x^3+\left(-1+x\right)\,\left(\lim_{h\rightarrow 0}{\frac{  -2\,x^3+2\,\left(h+x\right)^3}{h}}\right)}{\left(-1+x\right)^2}=  \frac{6\,\left(-1+x\right)\,x^2-2\,x^3}{\left(-1+x\right)^2}\]\textgreater\$expand((6*(x-1)*x\textsuperscript{2-2*x}3)/(x-1)\^{}2)

\[-\frac{6\,x^2}{1-2\,x+x^2}+\frac{4\,x^3}{1-2\,x+x^2}\]\textgreater function u(x)\&= 2*x\^{}3/(x-1)

\begin{verbatim}
                                   3
                                2 x
                               -------
                               - 1 + x
\end{verbatim}

\textgreater function h(x) \&= (6*(x-1)*x\textsuperscript{2-2*x}3)/(x-1)\^{}2

\begin{verbatim}
                                     2      3
                        6 (- 1 + x) x  - 2 x
                        ---------------------
                                      2
                             (- 1 + x)
\end{verbatim}

\textgreater plot2d(``u(x)'',``h(x)'',5.0):

\begin{figure}
\centering
\pandocbounded{\includegraphics[keepaspectratio]{images/KALKULUS_Diva Nagita(23030630024)-112.png}}
\caption{images/KALKULUS\_Diva\%20Nagita(23030630024)-112.png}
\end{figure}

\begin{enumerate}
\def\labelenumi{\arabic{enumi}.}
\setcounter{enumi}{4}
\item
  Aturan rantai

  Contoh:

  Terdapat f(x)=x\^{}2 dan g(x)=2x tentukan nilai dari (f(g(x)))'

  Kita bisa menentukan terlebih dahulu f(g(x)) yaitu

  f(g(x))=(2x)\^{}2

  f(g(x))=4x\^{}2

  Sehingga dapat kita gunakan aturan rantai sebagai berikut
\end{enumerate}

\textgreater\$showev(('limit((4*(x+h)\textsuperscript{2-4*x}2)/h,h,0))*('limit((2*(x+h)-2*x)/h,h,0)))

\[\left(\lim_{h\rightarrow 0}{\frac{-2\,x+2\,\left(h+x\right)}{h}}  \right)\,\left(\lim_{h\rightarrow 0}{\frac{-4\,x^2+4\,\left(h+x  \right)^2}{h}}\right)=16\,x\]6. Turunan dari Konstanta

Contoh:

Cari nilai turunan dari 1000

\textgreater\$showev('limit((1000-1000)/h,h,0))

\[0=0\]Dalam mengetahui cara kerja limit untuk mencari turunan akan diberikan beberapa contoh antara lain:

Pada contoh pertama akan dicari turunan

\[\ x^2\\]\textgreater\$showev('limit(((x+h)\textsuperscript{2-x}2)/h,h,0))

\[\lim_{h\rightarrow 0}{\frac{-x^2+\left(h+x\right)^2}{h}}=2\,x\]pertama-tama bisa kita uraikan operasi pada pembilang

\[\ (x+h)^2-x^2\\]

menguraikan

\[\ (x+h)^2\\]

menjadi

\[\ x^2+2hx+h^2\\]

kemudian disederhanakan dengan menghilangkan x\^{}2

\[\ x^2+2hx+h^2-x^2\]

menjadi

\[\ 2hx+h^2\\]\textgreater p \&= expand((x+h)\textsuperscript{2-x}2)\textbar simplify; \$p

\[h^2+2\,h\,x\]Kemudian bentuk pecahan

\[\frac{2hx+h^2}{h}\]

dapat disederhanakan menjadi

\[\ 2x+h\]\textgreater{} q \&=ratsimp(p/h); \$q

\[h+2\,x\]Setelah itu bisa dimasukkan nilai limit sebagai turunan dimana h mendekati 0 sehingga didapat turunan dari

\[\ x^2\\]

adalah

\[\ 2x\\]\textgreater\$limit(q,h,0)

\[2\,x\]Pada contoh kedua akan menentukan turunan dari

\[\ x^n\\]\textgreater\$showev('limit(((x+h)\textsuperscript{n-x}n)/h,h,0))

\[\lim_{h\rightarrow 0}{\frac{-x^{n}+\left(h+x\right)^{n}}{h}}=n\,x^{  -1+n}\]Pada contoh ke 3 menentukan turunan dari sin(x)

\textgreater\$showev('limit((sin(x+h)-sin(x))/h,h,0))

\[\lim_{h\rightarrow 0}{\frac{-\sin x+\sin \left(h+x\right)}{h}}=  \cos x\]Pada contoh berikutnya akan dicari turunan dari e\^{}x

namun maxima bermasalah dengan limit:

\[\lim_{h\to 0}\frac{e^{x+h}-e^x}{h}.\]Oleh karena itu perlu mengubah bentuk fungsi terlebih dahulu agar bisa mendapat hasil yang benar

\textgreater\$showev('limit((E\^{}h-1)/h,h,0))

\[\lim_{h\rightarrow 0}{\frac{-1+e^{h}}{h}}=1\]dengan memfaktorkan pembilang menjadi:

\textgreater\$showev('factor(E\textsuperscript{(x+h)-E}x))

\[{\it factor}\left(-e^{x}+e^{h+x}\right)=\left(-1+e^{h}\right)\,e^{x  }\]kemudian bisa kita masukkan hasil faktor tersebut kedalam limit, pada contoh, e\^{}x merupakan konstanta terhadap limit sehingga bisa dikeluarkan

\textgreater\$showev('limit(factor((E\textsuperscript{(x+h)-E}x)/h),h,0)) // turunan f(x)=e\^{}x

\[\left(\lim_{h\rightarrow 0}{\frac{-1+e^{h}}{h}}\right)\,e^{x}=e^{x}\]berikut adalah beberapa contoh penggunaan limit pada turunan fungsi

\begin{enumerate}
\def\labelenumi{\arabic{enumi}.}
\tightlist
\item
  turunan dari 1/x
\end{enumerate}

\textgreater\$showev('limit((1/(x+h)-1/x)/h,h,0))

\[\lim_{h\rightarrow 0}{\frac{-\frac{1}{x}+\frac{1}{h+x}}{h}}=-\frac{  1}{x^2}\]2. Turunan dari log(x)

\textgreater\$showev('limit((log(x+h)-log(x))/h,h,0))

\[\lim_{h\rightarrow 0}{\frac{-\log x+\log \left(h+x\right)}{h}}=  \frac{1}{x}\]3. Turunan dari tan(x)

\textgreater\$showev('limit((tan(x+h)-tan(x))/h,h,0))

\[\lim_{h\rightarrow 0}{\frac{-\tan x+\tan \left(h+x\right)}{h}}=  \frac{1}{\cos ^2x}\]4. Turunan dari arcsin(x)

\textgreater\$showev('limit((asin(x+h)-asin(x))/h,h,0))

\[\lim_{h\rightarrow 0}{\frac{-\arcsin x+\arcsin \left(h+x\right)}{h}  }=\frac{1}{\sqrt{1-x^2}}\]\# Turunan Fungsi

Definisi turunan:

\[f'(x) = \lim_{h\to 0} \frac{f(x+h)-f(x)}{h}\]Berikut adalah contoh-contoh menentukan turunan fungsi dengan menggunakan definisi turunan (limit).

\textgreater\$showev('limit(((x+h)\textsuperscript{2-x}2)/h,h,0)) // turunan x\^{}2

\[\lim_{h\rightarrow 0}{\frac{-x^2+\left(h+x\right)^2}{h}}=2\,x\]\textgreater p \&= expand((x+h)\textsuperscript{2-x}2)\textbar simplify; \$p //pembilang dijabarkan dan disederhanakan

\[h^2+2\,h\,x\]\textgreater q \&=ratsimp(p/h); \$q // ekspresi yang akan dihitung limitnya disederhanakan

\[h+2\,x\]\textgreater\$limit(q,h,0) // nilai limit sebagai turunan

\[2\,x\]\textgreater\$showev('limit(((x+h)\textsuperscript{n-x}n)/h,h,0)) // turunan x\^{}n

\[\lim_{h\rightarrow 0}{\frac{-x^{n}+\left(h+x\right)^{n}}{h}}=n\,x^{  -1+n}\]\textgreater\$showev('limit((sin(x+h)-sin(x))/h,h,0)) // turunan sin(x)

\[\lim_{h\rightarrow 0}{\frac{-\sin x+\sin \left(h+x\right)}{h}}=  \cos x\]\textgreater\$showev('limit((log(x+h)-log(x))/h,h,0)) // turunan log(x)

\[\lim_{h\rightarrow 0}{\frac{-\log x+\log \left(h+x\right)}{h}}=  \frac{1}{x}\]\textgreater\$showev('limit((1/(x+h)-1/x)/h,h,0)) // turunan 1/x

\[\lim_{h\rightarrow 0}{\frac{-\frac{1}{x}+\frac{1}{h+x}}{h}}=-\frac{  1}{x^2}\]\textgreater\$showev('limit((E\textsuperscript{(x+h)-E}x)/h,h,0)) // turunan f(x)=e\^{}x

\begin{verbatim}
Answering "Is x an integer?" with "integer"
Answering "Is x an integer?" with "integer"
Answering "Is x an integer?" with "integer"
Answering "Is x an integer?" with "integer"
Answering "Is x an integer?" with "integer"
Maxima is asking
Acceptable answers are: yes, y, Y, no, n, N, unknown, uk
Is x an integer?

Use assume!
Error in:
 $showev('limit((E^(x+h)-E^x)/h,h,0)) // turunan f(x)=e^x ...
                                     ^
\end{verbatim}

Maxima bermasalah dengan limit:

\[\lim_{h\to 0}\frac{e^{x+h}-e^x}{h}.\]Oleh karena itu diperlukan trik khusus agar hasilnya benar.

\textgreater\$showev('limit((E\^{}h-1)/h,h,0))

\[\lim_{h\rightarrow 0}{\frac{-1+e^{h}}{h}}=1\]\textgreater\$showev('factor(E\textsuperscript{(x+h)-E}x))

\[{\it factor}\left(-e^{x}+e^{h+x}\right)=\left(-1+e^{h}\right)\,e^{x  }\]\textgreater\$showev('limit(factor((E\textsuperscript{(x+h)-E}x)/h),h,0)) // turunan f(x)=e\^{}x

\[\left(\lim_{h\rightarrow 0}{\frac{-1+e^{h}}{h}}\right)\,e^{x}=e^{x}\]\textgreater function f(x) \&= x\^{}x

\begin{verbatim}
                                   x
                                  x
\end{verbatim}

\textgreater\$showev('limit(f(x),x,0))

\[\lim_{x\rightarrow 0}{x^{x}}=1\]\textgreater\$showev('limit((f(x+h)-f(x))/h,h,0)) // turunan f(x)=x\^{}x

\[\lim_{h\rightarrow 0}{\frac{-x^{x}+\left(h+x\right)^{h+x}}{h}}=  {\it infinity}\]Di sini Maxima juga bermasalah terkait limit:

\[\lim_{h\to 0} \frac{(x+h)^{x+h}-x^x}{h}.\]Dalam hal ini diperlukan asumsi nilai x.

\textgreater\&assume(x\textgreater0); \$showev('limit((f(x+h)-f(x))/h,h,0)) // turunan f(x)=x\^{}x

\[\lim_{h\rightarrow 0}{\frac{-x^{x}+\left(h+x\right)^{h+x}}{h}}=x^{x  }\,\left(1+\log x\right)\]\textgreater\&forget(x\textgreater0) // jangan lupa, lupakan asumsi untuk kembali ke semula

\begin{verbatim}
                               [x &gt; 0]
\end{verbatim}

\textgreater\&forget(x\textless0)

\begin{verbatim}
                               [x &lt; 0]
\end{verbatim}

\textgreater\&facts()

\begin{verbatim}
                            [t &gt; 0, r &gt; 0]
\end{verbatim}

\textgreater\$showev('limit((asin(x+h)-asin(x))/h,h,0)) // turunan arcsin(x)

\[\lim_{h\rightarrow 0}{\frac{-\arcsin x+\arcsin \left(h+x\right)}{h}  }=\frac{1}{\sqrt{1-x^2}}\]\textgreater\$showev('limit((tan(x+h)-tan(x))/h,h,0)) // turunan tan(x)

\[\lim_{h\rightarrow 0}{\frac{-\tan x+\tan \left(h+x\right)}{h}}=  \frac{1}{\cos ^2x}\]\textgreater function f(x) \&= sinh(x) // definisikan f(x)=sinh(x)

\begin{verbatim}
                               sinh(x)
\end{verbatim}

\textgreater function df(x) \&= limit((f(x+h)-f(x))/h,h,0); \$df(x) // df(x) = f'(x)

\[\frac{e^ {- x }\,\left(1+e^{2\,x}\right)}{2}\]Hasilnya adalah cosh(x), karena

\[\frac{e^x+e^{-x}}{2}=\cosh(x).\]\textgreater plot2d({[}``f(x)'',``df(x)''{]},-pi,pi,color={[}blue,red{]}):

\begin{figure}
\centering
\pandocbounded{\includegraphics[keepaspectratio]{images/KALKULUS_Diva Nagita(23030630024)-161.png}}
\caption{images/KALKULUS\_Diva\%20Nagita(23030630024)-161.png}
\end{figure}

\textgreater function f(x) \&= sin(3*x\textsuperscript{5+7)}2

\begin{verbatim}
                               2        5
                            sin (7 + 3 x )
\end{verbatim}

\textgreater diff(f,3), diffc(f,3)

\begin{verbatim}
1198.32948904
1198.72863721
\end{verbatim}

\textgreater\$showev('diff(f(x),x))

\[\frac{d}{d\,x}\,\sin ^2\left(7+3\,x^5\right)=30\,x^4\,\cos \left(7+  3\,x^5\right)\,\sin \left(7+3\,x^5\right)\]\textgreater\$\% with x=3

\[{\it \%at}\left(\frac{d}{d\,x}\,\sin ^2\left(7+3\,x^5\right) , x=3  \right)=2430\,\cos 736\,\sin 736\]\textgreater\$float(\%)

\[{\it \%at}\left(\frac{d^{1.0}}{d\,x^{1.0}}\,\sin ^2\left(7.0+3.0\,x  ^5\right) , x=3.0\right)=1198.728637211748\]\textgreater plot2d(f,0,3.1):

\begin{figure}
\centering
\pandocbounded{\includegraphics[keepaspectratio]{images/KALKULUS_Diva Nagita(23030630024)-165.png}}
\caption{images/KALKULUS\_Diva\%20Nagita(23030630024)-165.png}
\end{figure}

\textgreater function f(x) \&=5*cos(2*x)-2*x*sin(2*x) // mendifinisikan fungsi f

\begin{verbatim}
                      5 cos(2 x) - 2 x sin(2 x)
\end{verbatim}

\textgreater function df(x) \&=diff(f(x),x) // fd(x) = f'(x)

\begin{verbatim}
                     - 4 x cos(2 x) - 12 sin(2 x)
\end{verbatim}

\textgreater\$'f(1)=f(1), \$float(f(1)), \$'f(2)=f(2), \$float(f(2)) // nilai f(1) dan f(2)

\[-0.2410081230863468\]\pandocbounded{\includegraphics[keepaspectratio]{images/KALKULUS_Diva Nagita(23030630024)-167.png}}

\begin{figure}
\centering
\pandocbounded{\includegraphics[keepaspectratio]{images/KALKULUS_Diva Nagita(23030630024)-168.png}}
\caption{images/KALKULUS\_Diva\%20Nagita(23030630024)-168.png}
\end{figure}

\begin{figure}
\centering
\pandocbounded{\includegraphics[keepaspectratio]{images/KALKULUS_Diva Nagita(23030630024)-169.png}}
\caption{images/KALKULUS\_Diva\%20Nagita(23030630024)-169.png}
\end{figure}

\textgreater xp=solve(``df(x)'',1,2,0) // solusi f'(x)=0 pada interval {[}1, 2{]}

\begin{verbatim}
1.35822987384
\end{verbatim}

\textgreater df(xp), f(xp) // cek bahwa f'(xp)=0 dan nilai ekstrim di titik tersebut

\begin{verbatim}
0
-5.67530133759
\end{verbatim}

\textgreater plot2d({[}``f(x)'',``df(x)''{]},0,2*pi,color={[}blue,red{]}): //grafik fungsi dan turunannya

\begin{figure}
\centering
\pandocbounded{\includegraphics[keepaspectratio]{images/KALKULUS_Diva Nagita(23030630024)-170.png}}
\caption{images/KALKULUS\_Diva\%20Nagita(23030630024)-170.png}
\end{figure}

\section{Visualisasi Grafik dan Penggunaan Turunan Fungsi}\label{visualisasi-grafik-dan-penggunaan-turunan-fungsi}

Penggunaan Turunan fungsi antara lain

Menentukan Gradien Garis Singgung

Dimana jika terdapat fungsi y=f(x), maka gradien pada titik tertentu dapat dinyatakan sebagai m=f'(x)

sebagai contoh, kita akan cari garis singgung dari suatu grafik fungsi

\[\ x^2-2x+1\\]\textgreater function f(x) \&= x\^{}2-2*x+1

\begin{verbatim}
                                        2
                             1 - 2 x + x
\end{verbatim}

\textgreater function df(x) \&=diff(f(x),x) // fd(x) = f'(x)

\begin{verbatim}
                              - 2 + 2 x
\end{verbatim}

Setelah mengetahui turunan fungsi tersebut, kita coba memasukkan nilai x=2 maka didapat

\[\ m= 2(2)-2\\]\[\ m=2\\]kemudian kita masukkan nilai x pada fungsi awal untuk mendapat titik y

yaitu

\[\ f(2)=2^2-2.2+1\\]\[\ f(2)=1\\]didapat titik singgung pada grafik adalah (2,1) dengan gradien garis m=2

yang kemudian dapat kita cari persamaan garisnya:

\[\ y-y_1=m(x-x_1)\\]\[\ y-1=2(x-2)\\]\[\ y-1=2x-4\\]\[\ y=2x-3\\]yang kemudian dapat kita visualisasikan grafiknya

\textgreater function g(x) \&= 2*x-3

\begin{verbatim}
                              - 3 + 2 x
\end{verbatim}

\textgreater plot2d({[}``f(x)'',``g(x)''{]},color={[}blue,red{]}):

\begin{figure}
\centering
\pandocbounded{\includegraphics[keepaspectratio]{images/KALKULUS_Diva Nagita(23030630024)-180.png}}
\caption{images/KALKULUS\_Diva\%20Nagita(23030630024)-180.png}
\end{figure}

\begin{center}\rule{0.5\linewidth}{0.5pt}\end{center}

Pada turunan pertama fungsi dapat kita gunakan pula untuk menentukan titik kritis suatu fungsi dimana

\[\ f'(x)=a\\]

a adalah absis

Pada turunan kedua fungsi juga dapat kita gunakan untuk menentukan titik maksimum lokal atau titik minimum lokal suatu fungsi dimana c adalah titik kritis

\[\ f"(c)>0\\]

maka c titik minimum lokal

\[\ f"(c)<0\\]

maka c titik maksimum lokal

\begin{center}\rule{0.5\linewidth}{0.5pt}\end{center}

Berikut adalah contoh lain yaitu fungsi

\[\ f(x)=5cos(2x)-2sin(2x)\\]\textgreater function f(x) \&=5*cos(2*x)-2*x*sin(2*x)

\begin{verbatim}
                      5 cos(2 x) - 2 x sin(2 x)
\end{verbatim}

\textgreater function df(x) \&=diff(f(x),x) // fd(x) = f'(x)

\begin{verbatim}
                     - 4 x cos(2 x) - 12 sin(2 x)
\end{verbatim}

nilai f(1) dan f(2)

\textgreater\$'f(1)=f(1), \$float(f(1)), \$'f(2)=f(2), \$float(f(2))

\[-0.2410081230863468\]\pandocbounded{\includegraphics[keepaspectratio]{images/KALKULUS_Diva Nagita(23030630024)-186.png}}

\begin{figure}
\centering
\pandocbounded{\includegraphics[keepaspectratio]{images/KALKULUS_Diva Nagita(23030630024)-187.png}}
\caption{images/KALKULUS\_Diva\%20Nagita(23030630024)-187.png}
\end{figure}

\begin{figure}
\centering
\pandocbounded{\includegraphics[keepaspectratio]{images/KALKULUS_Diva Nagita(23030630024)-188.png}}
\caption{images/KALKULUS\_Diva\%20Nagita(23030630024)-188.png}
\end{figure}

solusi f'(x)=0 pada interval {[}1, 2{]}

\textgreater xp=solve(``df(x)'',1,2,0)

\begin{verbatim}
1.35822987384
\end{verbatim}

cek bahwa f'(xp)=0 dan nilai ekstrim di titik tersebut

\textgreater df(xp), f(xp)

\begin{verbatim}
0
-5.67530133759
\end{verbatim}

grafik fungsi dan turunannya

\textgreater plot2d({[}``f(x)'',``df(x)''{]},0,2*pi,color={[}blue,red{]}):

\begin{figure}
\centering
\pandocbounded{\includegraphics[keepaspectratio]{images/KALKULUS_Diva Nagita(23030630024)-189.png}}
\caption{images/KALKULUS\_Diva\%20Nagita(23030630024)-189.png}
\end{figure}

Contoh lain yaitu pada fungsi

\[\ f(x)=sinh(x)\\]\textgreater function f(x) \&= sinh(x)

\begin{verbatim}
                               sinh(x)
\end{verbatim}

\textgreater function df(x) \&= limit((f(x+h)-f(x))/h,h,0); \$df(x) // df(x) = f'(x)

\[\frac{e^ {- x }\,\left(1+e^{2\,x}\right)}{2}\]Didapat hasilnya adalah cosh(x), karena

\[\frac{e^x+e^{-x}}{2}=\cosh(x).\]\textgreater plot2d({[}``f(x)'',``df(x)''{]},-pi,pi,color={[}blue,red{]}):

\begin{figure}
\centering
\pandocbounded{\includegraphics[keepaspectratio]{images/KALKULUS_Diva Nagita(23030630024)-193.png}}
\caption{images/KALKULUS\_Diva\%20Nagita(23030630024)-193.png}
\end{figure}

\chapter{Integral}\label{integral}

\section{Jumlah Riemann (Riemann Sum)}\label{jumlah-riemann-riemann-sum}

Konsep dasar dari integral adalah menghitung luas di bawah kurva suatu fungsi. Untuk mendekati luas ini secara numerik, kita dapat membagi wilayah di bawah kurva menjadi persegi panjang-persegi panjang kecil, dan kemudian menjumlahkan luasnya. Ini disebut jumlah Riemann, yang menggunakan notasi sigma

\textgreater function f(x):=x\^{}2+1

\textgreater x=0:0.1:pi-0.1; plot2d(x,f(x+0.1),\textgreater bar); plot2d(``f(x)'',0,pi,\textgreater add):

\begin{figure}
\centering
\pandocbounded{\includegraphics[keepaspectratio]{images/KALKULUS_Diva Nagita(23030630024)-194.png}}
\caption{images/KALKULUS\_Diva\%20Nagita(23030630024)-194.png}
\end{figure}

Misalkan kita ingin menghitung luas di bawah kurva fungsi

f(x) dari x=a hingga x=b. Untuk mendekati luas ini, kita membagi interval{[}a,b{]} menjadi n subinterval yang sama panjang, di mana panjang setiap subinterval adalah:

\[\delta x= \frac{b-a}{n}\]Kemudian kita pilih titik xi dalam setiap subinterval sebagai titik representatif. Nilai f(xi) digunakan untuk menentukan tinggi persegi panjang, dan delta x sebagai lebarnya.

Jumlah Riemann didefinisikan sebagai:

image: Rumus\_Reimann

\section{Hubungan dengan Integral}\label{hubungan-dengan-integral}

etika kita meningkatkan jumlah subinterval (yakni

n menuju tak hingga), maka pendekatan jumlah Riemann ini semakin mendekati nilai integral. Dengan kata lain, integral tentu dari fungsi

f(x) dari a hingga b didefinisikan sebagai limit dari jumlah Riemann:

image: Rumus\_Hubungan\_Integral

Syarat integral

\section{Fungsi Kontinu}\label{fungsi-kontinu}

Secara matematis, fungsi f(x) dikatakan kontinu pada interval{[}a,b{]} jika memenuhi syarat-syarat berikut:

Definisi Kekontinuan fungsi(x) kontinu pada interval {[}a,b{]} jika: + Terdefinisi: Fungsi harus terdefinisi di setiap titik dalam interval tersebut. Artinya, untuk setiap nilai x dalam interval{[}a,b{]},f(x) harus memiliki nilai yang terdefinisi. + Limit Ada: Limit fungsi harus ada pada setiap titik dalam interval. + Untuk setiap titik dalam interval {[}a,b{]}, limit x=\textgreater c f(x) harus ada. + Limit Sama dengan Nilai Fungsi: Nilai fungsi pada titik tersebut harus sama dengan limit fungsi ketika mendekati titik tersebut. Yaitu, lim=\textgreater c f(x)=f(c).

Diketahui sebuah fungsi

\[h(x)=1/x\]

dari x=-2 hingga x=2

\textgreater function h(x):=1/x

\textgreater h(-2)

\begin{verbatim}
-0.5
\end{verbatim}

\textgreater h(-1)

\begin{verbatim}
-1
\end{verbatim}

\textgreater h(0)

\begin{verbatim}
Floating point error!
Try "trace errors" to inspect local variables after errors.
h:
    useglobal; return 1/x  
Error in:
h(0) ...
    ^
\end{verbatim}

Diketahui fungsi f(x)

\[f(x)=2x^3-3x+1\]

dari x=1 hingga x=2

\textgreater function f(x):=2*x\^{}3-3*x+1

\textgreater f(1)

\begin{verbatim}
0
\end{verbatim}

\textgreater f(2)

\begin{verbatim}
11
\end{verbatim}

\textgreater\$limit(2*x\^{}3-3*x+1,x,1)

\[0\]\textgreater\$limit(2*x\^{}3-3*x+1,x,2)

\[11\]\# Integral

EMT dapat digunakan untuk menghitung integral, baik integral tak tentu maupun integral tentu. Untuk integral tak tentu (simbolik) sudah tentu EMT menggunakan Maxima, sedangkan untuk perhitungan integral tentu EMT sudah menyediakan beberapa fungsi yang mengimplementasikan algoritma kuadratur (perhitungan integral tentu menggunakan metode numerik).

Pada notebook ini akan ditunjukkan perhitungan integral tentu dengan menggunakan Teorema Dasar Kalkulus:

\[\int_a^b f(x)\ dx = F(b)-F(a), \quad \text{ dengan  } F'(x) = f(x).\]Fungsi untuk menentukan integral adalah integrate. Fungsi ini dapat digunakan untuk menentukan, baik integral tentu maupun tak tentu (jika fungsinya memiliki antiderivatif). Untuk perhitungan integral tentu fungsi integrate menggunakan metode numerik (kecuali fungsinya tidak integrabel, kita tidak akan menggunakan metode ini).

\section{Integral Tak Tentu}\label{integral-tak-tentu}

EMT dapat digunakan untuk menghitung integral, baik integral tak tentu maupun integral tentu. Untuk integral tak tentu (simbolik) sudah tentu EMT menggunakan Maxima.

Pada notebook ini akan ditunjukkan perhitungan integral tentu dengan menggunakan Teorema Dasar Kalkulus:

\[\int f(x)\ dx = F(x)+C\]Fungsi untuk menentukan integral adalah integrate. Fungsi ini dapat digunakan untuk menentukan, baik integral tentu maupun tak tentu (jika fungsinya memiliki antiderivatif).

\textgreater function f(x):=(1/x\^{}2+1)

\textgreater x=0:0.1:pi-0.1; plot2d(x,f(x+0.1),\textgreater bar); plot2d(``f(x)'',0,pi,\textgreater add):

\begin{figure}
\centering
\pandocbounded{\includegraphics[keepaspectratio]{images/KALKULUS_Diva Nagita(23030630024)-202.png}}
\caption{images/KALKULUS\_Diva\%20Nagita(23030630024)-202.png}
\end{figure}

\textgreater\$showev('integrate(x\^{}n,x))

\begin{verbatim}
Answering "Is n equal to -1?" with "no"
\end{verbatim}

\[\int {x^{n}}{\;dx}=\frac{x^{1+n}}{1+n}\]\textgreater\$showev('integrate(1/(1+x\^{}2),x))

\[\int {\frac{1}{1+x^2}}{\;dx}=\arctan x\]\textgreater\$showev('integrate(1/sqrt(1-x\^{}2),x))

\[\int {\frac{1}{\sqrt{1-x^2}}}{\;dx}=\arcsin x\]\textgreater\$showev('integrate(sin(x),x,0,pi))

\[\int_{0}^{\pi}{\sin x\;dx}=2\]\textgreater plot2d(``sin(x)'',0,2*pi):

\begin{figure}
\centering
\pandocbounded{\includegraphics[keepaspectratio]{images/KALKULUS_Diva Nagita(23030630024)-207.png}}
\caption{images/KALKULUS\_Diva\%20Nagita(23030630024)-207.png}
\end{figure}

\textgreater\$showev('integrate((y\^{}2),y))

\[\int {y^2}{\;dy}=\frac{y^3}{3}\]\textgreater function f(x):=(x\textsuperscript{5+3*x}2+x+3)

\textgreater x=0:0.1:pi-0.1; plot2d(x,f(x+0.1),\textgreater bar); plot2d(``f(x)'',0,pi,\textgreater add):

\begin{figure}
\centering
\pandocbounded{\includegraphics[keepaspectratio]{images/KALKULUS_Diva Nagita(23030630024)-209.png}}
\caption{images/KALKULUS\_Diva\%20Nagita(23030630024)-209.png}
\end{figure}

\textgreater\$showev('integrate(sin(x),x,a,b))

\[\int_{a}^{b}{\sin x\;dx}=\cos a-\cos b\]\textgreater\$showev('integrate(x\^{}n,x,a,b))

\begin{verbatim}
Answering "Is n positive, negative or zero?" with "positive"
\end{verbatim}

\[\int_{a}^{b}{x^{n}\;dx}=-\frac{a^{1+n}}{1+n}+\frac{b^{1+n}}{1+n}\]\textgreater\$showev('integrate(x\^{}2*sqrt(2*x+1),x,0,2))

\[\int_{0}^{2}{x^2\,\sqrt{1+2\,x}\;dx}=-\frac{2}{105}+\frac{2\,5^{  \frac{5}{2}}}{21}\]\textgreater\$ratsimp(\%)

\[\int_{0}^{2}{x^2\,\sqrt{1+2\,x}\;dx}=\frac{-2+2\,5^{\frac{7}{2}}}{  105}\]\textgreater\$showev('integrate(x\textsuperscript{5+3*x}2+x+3, x))

\[\int {3+x+3\,x^2+x^5}{\;dx}=3\,x+\frac{x^2}{2}+x^3+\frac{x^6}{6}\]\textgreater\$showev('integrate(3*x\^{}3-2*x,x))

\[\int {-2\,x+3\,x^3}{\;dx}=-x^2+\frac{3\,x^4}{4}\]\textgreater\$showev('integrate((sin(sqrt(x)+a)*E\textsuperscript{sqrt(x))/sqrt(x),x,0,pi}2))

\[\int_{0}^{\pi^2}{\frac{\sin \left(a+\sqrt{x}\right)\,e^{\sqrt{x}}}{  \sqrt{x}}\;dx}=\left(1+e^{\pi}\right)\,\cos a+\left(-1-e^{\pi}  \right)\,\sin a\]\textgreater\$factor(\%)

\[\int_{0}^{\pi^2}{\frac{\sin \left(a+\sqrt{x}\right)\,e^{\sqrt{x}}}{  \sqrt{x}}\;dx}=\left(-1-e^{\pi}\right)\,\left(-\cos a+\sin a\right)\]\textgreater\$showev('integrate(x\^{}2*sqrt(2*x+1),x))

\[\int {x^2\,\sqrt{1+2\,x}}{\;dx}=\frac{\left(1+2\,x\right)^{\frac{3  }{2}}}{12}-\frac{\left(1+2\,x\right)^{\frac{5}{2}}}{10}+\frac{\left(  1+2\,x\right)^{\frac{7}{2}}}{28}\]\textgreater\$showev('integrate (cos(2*x),x))

\[\int {\cos \left(2\,x\right)}{\;dx}=\frac{\sin \left(2\,x\right)}{2  }\]\textgreater function map f(x) \&= E\textsuperscript{(-x}2)

\begin{verbatim}
                                    2
                                 - x
                                E
\end{verbatim}

\textgreater\$showev('integrate(f(x),x))

\[\int {e^ {- x^2 }}{\;dx}=\frac{\sqrt{\pi}\,\mathrm{erf}\left(x  \right)}{2}\]Fungsi f tidak memiliki antiturunan, integralnya masih memuat integral lain.

Fungsi error, dinotasikan sebagai erf(x), didefinisikan sebagai integral tertentu berikut:

\[erf(x) = \int \frac{e^{-x^2}}{\sqrt{\pi}} \ dx.\]\#\# Integral tentu

Integral tentu adalah integral matematika yang memiliki batasan atas dan bawah yang jelas, sehingga menghasilkan sebuah nilai. Batasan dari integral tentu adalah a sampai b atau batas atas sampai batas bawah.

Untuk perhitungan integral tentu fungsi integrate menggunakan metode numerik (kecuali fungsinya tidak integrabel, kita tidak akan menggunakan metode ini).

Kita tidak dapat menggunakan teorema Dasar kalkulus untuk menghitung integral tentu fungsi tersebut jika semua batasnya berhingga. Dalam hal ini dapat digunakan metode numerik (rumus kuadratur).

\[\int_a^b (f(x))dx\]\[F(b)-F(a)\]\[\int_{0}^{\infty }{e^ {- x^2 }\;dx}\]dapat dihampiri dengan jumlah luas persegi-persegi panjang di bawah kurva y=f(x) tersebut. Langkah-langkahnya adalah sebagai berikut.

\textgreater{} function map f(x) \&= E\textsuperscript{(-x}2)

\begin{verbatim}
                                    2
                                 - x
                                E
\end{verbatim}

\textgreater\$showev('integrate(f(x),x))

\[\int {e^ {- x^2 }}{\;dx}=\frac{\sqrt{\pi}\,\mathrm{erf}\left(x  \right)}{2}\]\textgreater x=0:0.1:pi-0.1; plot2d(x,f(x+0.1),\textgreater bar); plot2d(``f(x)'',0,pi,\textgreater add):

\begin{figure}
\centering
\pandocbounded{\includegraphics[keepaspectratio]{images/KALKULUS_Diva Nagita(23030630024)-226.png}}
\caption{images/KALKULUS\_Diva\%20Nagita(23030630024)-226.png}
\end{figure}

\textgreater t \&= makelist(a,a,0,pi-0.1,0.1); // t sebagai list untuk menyimpan nilai-nilai x

\textgreater fx \&= makelist(f(t{[}i{]}+0.1),i,1,length(t)); // simpan nilai-nilai f(x)

\textgreater\$showev('integrate(f(x),x,0,inf))

\[\int_{0}^{\infty }{e^ {- x^2 }\;dx}=\frac{\sqrt{\pi}}{2}\]* fx \&=: Ini adalah operasi penugasan gabungan. Artinya, hasil dari perhitungan di sebelah kanan akan ditambahkan ke variabel fx. Jika fx belum didefinisikan sebelumnya, maka fx akan dibuat sebagai sebuah list. * -makelist(\ldots,i,1,length(t)): Fungsi ini digunakan untuk membuat * sebuah list.

Parameter-parameternya adalah:

\begin{itemize}
\item
  f(t{[}i{]}+0.1): Ini adalah ekspresi yang akan dihitung untuk setiap nilai i. Fungsi f akan dievaluasi pada nilai t{[}i{]} ditambah 0.1.
\item
  i: Adalah indeks yang akan digunakan untuk mengakses elemen-elemen dalam list t.
\item
  1: Adalah nilai awal untuk indeks i.
\item
  length(t): Adalah nilai akhir untuk indeks i, yaitu panjang dari list t.
\end{itemize}

\textgreater\$showev('integrate(x*exp(-x),x,0,1))

\[\int_{0}^{1}{x\,e^ {- x }\;dx}=1-2\,e^ {- 1 }\]tentukan integral dari

\[3x^2+4x+2\]

darix=1 sampai x=2

\textgreater function f(x) \&= x\^{}x

\begin{verbatim}
                                   x
                                  x
\end{verbatim}

\textgreater\$showev('integrate(f(x),x,0,1))

\[\int_{0}^{1}{x^{x}\;dx}=\int_{0}^{1}{x^{x}\;dx}\]\textgreater x=0:0.1:1-0.01; plot2d(x,f(x+0.01),\textgreater bar); plot2d(``f(x)'',0,1,\textgreater add):

\begin{figure}
\centering
\pandocbounded{\includegraphics[keepaspectratio]{images/KALKULUS_Diva Nagita(23030630024)-231.png}}
\caption{images/KALKULUS\_Diva\%20Nagita(23030630024)-231.png}
\end{figure}

\textgreater function f(x) \&= sin(3*x\textsuperscript{5+7)}2

\begin{verbatim}
                               2        5
                            sin (7 + 3 x )
\end{verbatim}

\textgreater integrate(f,0,1)

\begin{verbatim}
0.542581176074
\end{verbatim}

\textgreater\&showev('integrate(f(x),x,0,1))

\begin{verbatim}
         1                           1              pi
        /                      gamma(-) sin(14) sin(--)
        [     2        5             5              10
        I  sin (7 + 3 x ) dx = ------------------------
        ]                                  1/5
        /                              10 6
         0
                4/5                    1
 - (- 60 + ((- 6    I gamma_incomplete(-, - 6 I)
                                       5
    4/5                    1
 + 6    I gamma_incomplete(-, 6 I)) cos(14)
                           5
     4/5                  1            4/5                  1
 + (6    gamma_incomplete(-, - 6 I) + 6    gamma_incomplete(-, 6 I))
                          5                                 5
              pi
 sin(14)) sin(--))/120
              10
\end{verbatim}

\textgreater\&float(\%)

\begin{verbatim}
         1.0
        /
        [       2            5
        I    sin (7.0 + 3.0 x ) dx = 
        ]
        /
         0.0
0.09820784258795788 - 0.008333333333333333
 (- 60.0 + 0.3090169943749474 (0.9906073556948704
 (4.192962712629476 gamma__incomplete(0.2, - 6.0 I)
 + 4.192962712629476 gamma__incomplete(0.2, 6.0 I))
 + 0.1367372182078336 (- 4.192962712629476 I
 gamma__incomplete(0.2, - 6.0 I) + 4.192962712629476 I
 gamma__incomplete(0.2, 6.0 I))))
\end{verbatim}

\textgreater\$showev('integrate(x*exp(-x),x,0,1)) // Integral tentu (eksak)

\[\int_{0}^{1}{x\,e^ {- x }\;dx}=1-2\,e^ {- 1 }\]\textgreater function h(x):=3*x\^{}2+4*x+2

\textgreater\$showev('integrate (3*x\^{}2+4*x+2,x,1,2))

\[\int_{1}^{2}{2+4\,x+3\,x^2\;dx}=15\]\textgreater\$showev('integrate (ln(x),x,1,2))

\[\int_{1}^{2}{\log x\;dx}=-1+2\,\log 2\]\textgreater\$showev('integrate (sin(x),x,0,pi))

\[\int_{0}^{\pi}{\sin x\;dx}=2\]\textgreater\$showev('integrate (cos(2*x),x))

\[\int {\cos \left(2\,x\right)}{\;dx}=\frac{\sin \left(2\,x\right)}{2  }\]\textgreater reset;

\section{Aplikasi Integral Tentu}\label{aplikasi-integral-tentu}

\textgreater plot2d(``x\^{}3-x'',-0.1,1.1); plot2d(``-x\^{}2'',\textgreater add); \ldots{}\\
\textgreater{} b=solve(``x\textsuperscript{3-x+x}2'',0.5); x=linspace(0,b,200); xi=flipx(x); \ldots{}\\
\textgreater{} plot2d(x\textbar xi,x\textsuperscript{3-x\textbar-xi}2,\textgreater filled,style=``\textbar{}'',fillcolor=1,\textgreater add): // Plot daerah antara 2 kurva

\begin{figure}
\centering
\pandocbounded{\includegraphics[keepaspectratio]{images/KALKULUS_Diva Nagita(23030630024)-237.png}}
\caption{images/KALKULUS\_Diva\%20Nagita(23030630024)-237.png}
\end{figure}

\textgreater a=solve(``x\textsuperscript{3-x+x}2'',0), b=solve(``x\textsuperscript{3-x+x}2'',1) // absis titik-titik potong kedua kurva

\begin{verbatim}
0
0.61803398875
\end{verbatim}

\textgreater integrate(``(-x\textsuperscript{2)-(x}3-x)'',a,b) // luas daerah yang diarsir

\begin{verbatim}
0.0758191713542
\end{verbatim}

\textgreater a \&= solve((-x\textsuperscript{2)-(x}3-x),x); \$a // menentukan absis titik potong kedua kurva secara eksak

\[\left[ x=\frac{-\sqrt{5}-1}{2} , x=\frac{\sqrt{5}-1}{2} , x=0
  \right] \]\textgreater\$showev('integrate(-x\textsuperscript{2-x}3+x,x,0,(sqrt(5)-1)/2)) // Nilai integral secara eksak

\[\int_{0}^{\frac{\sqrt{5}-1}{2}}{-x^3-x^2+x\;dx}=\frac{13-5^{\frac{3
 }{2}}}{24}\]\textgreater\$float(\%)

\[\int_{0.0}^{0.6180339887498949}{-1.0\,x^3-1.0\,x^2+x\;dx}=
 0.07581917135421037\]1. Panjang Kurva

Panjang kurva dari kurva

y=f(x) pada interval {[}a,b{]}dapat dihitung dengan menggunakan integral. Panjang kurva L dapat didefinisikan sebagai:

image: Rumus\_Panjang\_Kurva

Tentukan panjang kurva

\[y=x^2\]

dari x=0 hingga x=2

\textgreater\&diff(x\^{}2, x)

\begin{verbatim}
                                 2 x
\end{verbatim}

\textgreater\$showev('integrate (sqrt(1+(2*x)\^{}2), x, 0, 2))

\[\int_{0}^{2}{\sqrt{4\,x^2+1}\;dx}=\frac{{\rm asinh}\; 4+4\,\sqrt{17
 }}{4}\]Hitunglah panjang kurva berikut ini dan luas daerah di dalam kurva tersebut.

\[\gamma(t) = (r(t) \cos(t), r(t) \sin(t))\]dengan

\[r(t) = 1 + \dfrac{\sin(3t)}{2},\quad 0\le t\le 2\pi.\]\textgreater t=linspace(0,2pi,1000); r=1+sin(3*t)/2; x=r*cos(t); y=r*sin(t); \ldots{}\\
\textgreater{} plot2d(x,y,\textgreater filled,fillcolor=red,style=``/'',r=1.5): // Kita gambar kurvanya terlebih dahulu

\begin{figure}
\centering
\pandocbounded{\includegraphics[keepaspectratio]{images/KALKULUS_Diva Nagita(23030630024)-245.png}}
\caption{images/KALKULUS\_Diva\%20Nagita(23030630024)-245.png}
\end{figure}

\textgreater function r(t) \&= 1+sin(3*t)/2; \$'r(t)=r(t)

\[r\left(t\right)=\frac{\sin \left(3\,t\right)}{2}+1\]\textgreater function fx(t) \&= r(t)*cos(t); \$'fx(t)=fx(t)

\[{\it fx}\left(t\right)=\cos t\,\left(\frac{\sin \left(3\,t\right)}{
 2}+1\right)\]\textgreater function fy(t) \&= r(t)*sin(t); \$'fy(t)=fy(t):

\textgreater function ds(t) \&= trigreduce(radcan(sqrt(diff(fx(t),t)\textsuperscript{2+diff(fy(t),t)}2))); \$'ds(t)=ds(t):

\textgreater\$integrate(ds(x),x,0,2*pi) //panjang (keliling) kurva

\[\frac{\int_{0}^{2\,\pi}{\sqrt{4\,\cos \left(6\,x\right)+4\,\sin 
 \left(3\,x\right)+9}\;dx}}{2}\]\textgreater integrate(``ds(x)'',0,2*pi)

\begin{verbatim}
9.0749467823
\end{verbatim}

Spiral Logaritmik

\[x=e^{ax}\cos x,\ y=e^{ax}\sin x.\]\textgreater a=0.1; plot2d(``exp(a*x)*cos(x)'',``exp(a*x)*sin(x)'',r=2,xmin=0,xmax=2*pi):

\begin{figure}
\centering
\pandocbounded{\includegraphics[keepaspectratio]{images/KALKULUS_Diva Nagita(23030630024)-250.png}}
\caption{images/KALKULUS\_Diva\%20Nagita(23030630024)-250.png}
\end{figure}

\section{Koordinat Kartesius}\label{koordinat-kartesius}

Perhitungan panjang kurva menggunakan koordinat Kartesius. Kita akan hitung panjang kurva dengan persamaan implisit:

\[x^3+y^3-3xy=0.\]\textgreater z \&= x\textsuperscript{3+y}3-3*x*y; \$z

\[y^3-3\,x\,y+x^3\]\textgreater plot2d(z,r=2,level=0,n=100):

\begin{figure}
\centering
\pandocbounded{\includegraphics[keepaspectratio]{images/KALKULUS_Diva Nagita(23030630024)-253.png}}
\caption{images/KALKULUS\_Diva\%20Nagita(23030630024)-253.png}
\end{figure}

\textgreater plot2d(z,a=0,b=2,c=0,d=2,level={[}-10;0{]},n=100,contourwidth=3,style=``/''):

\begin{figure}
\centering
\pandocbounded{\includegraphics[keepaspectratio]{images/KALKULUS_Diva Nagita(23030630024)-254.png}}
\caption{images/KALKULUS\_Diva\%20Nagita(23030630024)-254.png}
\end{figure}

\textgreater\$z with y=l*x, sol \&= solve(\%,x); \$sol

\[l^3\,x^3+x^3-3\,l\,x^2\]\[\left[ x=\frac{3\,l}{l^3+1} , x=0 \right] \]\textgreater function f(l) \&= rhs(sol{[}1{]}); \$'f(l)=f(l)

\[f\left(l\right)=\frac{3\,l}{l^3+1}\]\textgreater plot2d(\&f(x),\&x*f(x),xmin=-0.5,xmax=2,a=0,b=2,c=0,d=2,r=1.5):

\begin{figure}
\centering
\pandocbounded{\includegraphics[keepaspectratio]{images/KALKULUS_Diva Nagita(23030630024)-258.png}}
\caption{images/KALKULUS\_Diva\%20Nagita(23030630024)-258.png}
\end{figure}

Elemen panjang kurva adalah:

\[ds=\sqrt{f'(l)^2+(lf'(l)+f(l))^2}.\]\textgreater function ds(l) \&= ratsimp(sqrt(diff(f(l),l)\textsuperscript{2+diff(l*f(l),l)}2)); \$'ds(l)=ds(l)

\[{\it ds}\left(l\right)=\frac{\sqrt{9\,l^8+36\,l^6-36\,l^5-36\,l^3+
 36\,l^2+9}}{\sqrt{l^{12}+4\,l^9+6\,l^6+4\,l^3+1}}\]\textgreater\$integrate(ds(l),l,0,1)

\[\int_{0}^{1}{\frac{\sqrt{9\,l^8+36\,l^6-36\,l^5-36\,l^3+36\,l^2+9}
 }{\sqrt{l^{12}+4\,l^9+6\,l^6+4\,l^3+1}}\;dl}\]Integral tersebut tidak dapat dihitung secara eksak menggunakan Maxima. Kita hitung integral etrsebut secara numerik dengan Euler. Karena kurva simetris, kita hitung untuk nilai variabel integrasi dari 0 sampai 1, kemudian hasilnya dikalikan 2.

\textgreater2*integrate(``ds(x)'',0,1)

\begin{verbatim}
4.91748872168
\end{verbatim}

\textgreater2*romberg(\&ds(x),0,1)// perintah Euler lain untuk menghitung nilai hampiran integral yang sama

\begin{verbatim}
4.91748872168
\end{verbatim}

\textgreater function panjangkurva(fx,fy,a,b)\ldots{}

\begin{verbatim}
ds=mxm("sqrt(diff(@fx,x)^2+diff(@fy,x))^2");
return romberg(ds,a,b);
endfunction
\end{verbatim}

\textgreater panjangkurva(``x'',``x\^{}2'',-1,1)

\begin{verbatim}
2
\end{verbatim}

\textgreater2*panjangkurva(mxm(``f(x)''),mxm(``x*f(x)''),0,1)

\begin{verbatim}
10.9351907861
\end{verbatim}

\begin{enumerate}
\def\labelenumi{\arabic{enumi}.}
\setcounter{enumi}{1}
\tightlist
\item
  Spiral Logaritmik
\end{enumerate}

\[x=e^{ax}\cos x,\ y=e^{ax}\sin x.\]\textgreater plot2d(``x*cos(x)'',``x*sin(x)'',xmin=0,xmax=2*pi,square=1):

\begin{figure}
\centering
\pandocbounded{\includegraphics[keepaspectratio]{images/KALKULUS_Diva Nagita(23030630024)-263.png}}
\caption{images/KALKULUS\_Diva\%20Nagita(23030630024)-263.png}
\end{figure}

\textgreater panjangkurva(``x*cos(x)'',``x*sin(x)'',0,2*pi)

\begin{verbatim}
46.0540912208
\end{verbatim}

Berikut kita definisikan fungsi yang sama namun dengan Maxima, untuk perhitungan eksak.

\textgreater\&kill(ds,x,fx,fy)

\begin{verbatim}
                                 done
\end{verbatim}

\textgreater function ds(fx,fy) \&\&= sqrt(diff(fx,x)\textsuperscript{2+diff(fy,x)}2)

\begin{verbatim}
                           2              2
                  sqrt(diff (fy, x) + diff (fx, x))
\end{verbatim}

\textgreater sol \&= ds(x*cos(x),x*sin(x)); \$sol // Kita gunakan untuk menghitung panjang kurva terakhir di atas

\[\sqrt{\left(\cos x-x\,\sin x\right)^2+\left(\sin x+x\,\cos x\right)
 ^2}\]\textgreater\$sol \textbar{} trigreduce \textbar{} expand, \$integrate(\%,x,0,2*pi), \%()

\[\sqrt{x^2+1}\]\[\frac{{\rm asinh}\; \left(2\,\pi\right)+2\,\pi\,\sqrt{4\,\pi^2+1}}{
 2}\] 21.2562941482

\textgreater plot2d(``3*x\textsuperscript{2-1'',''3*x}3-1'',xmin=-1/sqrt(3),xmax=1/sqrt(3),square=1):

\begin{figure}
\centering
\pandocbounded{\includegraphics[keepaspectratio]{images/KALKULUS_Diva Nagita(23030630024)-267.png}}
\caption{images/KALKULUS\_Diva\%20Nagita(23030630024)-267.png}
\end{figure}

\textgreater sol \&= radcan(ds(3*x\textsuperscript{2-1,3*x}3-1)); \$sol

\[3\,x\,\sqrt{9\,x^2+4}\]\textgreater\$showev('integrate(sol,x,0,1/sqrt(3))), \$2*float(\%) // panjang kurva di atas

\[3\,\int_{0}^{\frac{1}{\sqrt{3}}}{x\,\sqrt{9\,x^2+4}\;dx}=3\,\left(
 \frac{7^{\frac{3}{2}}}{27}-\frac{8}{27}\right)\]\[6.0\,\int_{0.0}^{0.5773502691896258}{x\,\sqrt{9.0\,x^2+4.0}\;dx}=
 2.337835372767141\]\textgreater a=0.1; plot2d(``exp(a*x)*cos(x)'',``exp(a*x)*sin(x)'',r=2,xmin=0,xmax=2*pi):

\begin{figure}
\centering
\pandocbounded{\includegraphics[keepaspectratio]{images/KALKULUS_Diva Nagita(23030630024)-271.png}}
\caption{images/KALKULUS\_Diva\%20Nagita(23030630024)-271.png}
\end{figure}

\textgreater\&kill(a) // hapus expresi a

\begin{verbatim}
                                 done
\end{verbatim}

\textgreater function fx(t) \&= exp(a*t)*cos(t); \$'fx(t)=fx(t)

\[{\it fx}\left(t\right)=e^{a\,t}\,\cos t\]\textgreater function fy(t) \&= exp(a*t)*sin(t); \$'fy(t)=fy(t)

\[{\it fy}\left(t\right)=e^{a\,t}\,\sin t\]\textgreater function df(t) \&= trigreduce(radcan(sqrt(diff(fx(t),t)\textsuperscript{2+diff(fy(t),t)}2))); \$'df(t)=df(t)

\textgreater S \&=integrate(df(t),t,0,2*\%pi); \$S

\[\sqrt{a^2+1}\,\left(\frac{e^{2\,\pi\,a}}{a}-\frac{1}{a}\right)\]\textgreater S(a=0.1) // Panjang kurva untuk a=0.1

\begin{verbatim}
8.78817491636
\end{verbatim}

\textgreater plot2d(``x\^{}2'',xmin=-1,xmax=1):

\begin{figure}
\centering
\pandocbounded{\includegraphics[keepaspectratio]{images/KALKULUS_Diva Nagita(23030630024)-275.png}}
\caption{images/KALKULUS\_Diva\%20Nagita(23030630024)-275.png}
\end{figure}

\textgreater\$showev('integrate(sqrt(1+diff(x\textsuperscript{2,x)}2),x,-1,1))

\[\int_{-1}^{1}{\sqrt{4\,x^2+1}\;dx}=\frac{{\rm asinh}\; 2+2\,\sqrt{5
 }}{2}\]\textgreater\$float(\%)

\[\int_{-1.0}^{1.0}{\sqrt{4.0\,x^2+1.0}\;dx}=2.957885715089195\]\textgreater x=-1:0.2:1; y=x\^{}2; plot2d(x,y); \ldots{}\\
\textgreater{} plot2d(x,y,points=1,style=``o\#'',add=1):

\begin{figure}
\centering
\pandocbounded{\includegraphics[keepaspectratio]{images/KALKULUS_Diva Nagita(23030630024)-278.png}}
\caption{images/KALKULUS\_Diva\%20Nagita(23030630024)-278.png}
\end{figure}

\textgreater i=1:cols(x)-1; sum(sqrt((x{[}i+1{]}-x{[}i{]})\textsuperscript{2+(y{[}i+1{]}-y{[}i{]})}2))

\begin{verbatim}
2.95191957027
\end{verbatim}

\section{Sikloid}\label{sikloid}

Berikut kita akan menghitung panjang kurva lintasan (sikloid) suatu titik pada lingkaran yang berputar ke kanan pada permukaan datar. Misalkan jari-jari lingkaran tersebut adalah r. Posisi titik pusat lingkaran pada saat t adalah:

\[(rt,r).\]Misalkan posisi titik pada lingkaran tersebut mula-mula (0,0) dan posisinya pada saat t adalah:

\[(r(t-\sin(t)),r(1-\cos(t))).\]Berikut kita plot lintasan tersebut dan beberapa posisi lingkaran ketika t=0, t=pi/2, t=r*pi.

\textgreater x \&= r*(t-sin(t))

\begin{verbatim}
        [0, 1.665833531718508e-4 r, 0.001330669204938795 r, 
0.004479793338660443 r, 0.0105816576913495 r, 0.02057446139579699 r, 
0.03535752660496461 r, 0.05578231276230894 r, 0.08264390910047725 r, 
0.1166730903725166 r, 0.1585290151921035 r, 0.2087926399385646 r, 
0.2679609140327737 r, 0.3364418145828071 r, 0.4145502700115399 r, 
0.5025050133959458 r, 0.6004263969584952 r, 0.7083351895475318 r, 
0.8261523691218055 r, 0.9536999123125863 r, 1.090702573174319 r, 
1.236790633351127 r, 1.391503596180411 r, 1.554294787823281 r, 
1.724536819448851 r, 1.901527855896045 r, 2.084498628178538 r, 
2.272620119766172 r, 2.465011849844097 r, 2.66075067078602 r, 
2.858879991940135 r]
\end{verbatim}

\textgreater y \&= r*(1-cos(t))

\begin{verbatim}
        [0, 0.004995834721974179 r, 0.01993342215875837 r, 
0.04466351087439402 r, 0.0789390059971149 r, 0.1224174381096272 r, 
0.1746643850903217 r, 0.2351578127155115 r, 0.3032932906528345 r, 
0.3783900317293355 r, 0.4596976941318602 r, 0.5464038785744225 r, 
0.6376422455233264 r, 0.7325011713754126 r, 0.8300328570997592 r, 
0.9292627983322973 r, 1.029199522301289 r, 1.128844494295525 r, 
1.227202094693087 r, 1.323289566863504 r, 1.416146836547143 r, 
1.504846104599858 r, 1.588501117255346 r, 1.666276021279825 r, 
1.737393715541246 r, 1.801143615546934 r, 1.856888753368948 r, 
1.904072142017062 r, 1.942222340668659 r, 1.970958165149591 r, 
1.989992496600446 r]
\end{verbatim}

Berikut kita gambar sikloid untuk r=1.

\textgreater ex \&= x-sin(x); ey \&= 1-cos(x); aspect(1);

\textgreater plot2d(ex,ey,xmin=0,xmax=4pi,square=1); \ldots{}\\
\textgreater{} plot2d(``2+cos(x)'',``1+sin(x)'',xmin=0,xmax=2pi,\textgreater add,color=blue); \ldots{}\\
\textgreater{} plot2d({[}2,ex(2){]},{[}1,ey(2){]},color=red,\textgreater add); \ldots{}\\
\textgreater{} plot2d(ex(2),ey(2),\textgreater points,\textgreater add,color=red); \ldots{}\\
\textgreater{} plot2d(``2pi+cos(x)'',``1+sin(x)'',xmin=0,xmax=2pi,\textgreater add,color=blue); \ldots{}\\
\textgreater{} plot2d({[}2pi,ex(2pi){]},{[}1,ey(2pi){]},color=red,\textgreater add); \ldots{}\\
\textgreater{} plot2d(ex(2pi),ey(2pi),\textgreater points,\textgreater add,color=red):

\begin{figure}
\centering
\pandocbounded{\includegraphics[keepaspectratio]{images/KALKULUS_Diva Nagita(23030630024)-281.png}}
\caption{images/KALKULUS\_Diva\%20Nagita(23030630024)-281.png}
\end{figure}

\textgreater ds \&= radcan(sqrt(diff(ex,x)\textsuperscript{2+diff(ey,x)}2)); \$ds=trigsimp(ds) // elemen panjang kurva sikloid

\[\sqrt{\sin ^2x+\cos ^2x-2\,\cos x+1}=\sqrt{2-2\,\cos x}\]\textgreater ds \&= trigsimp(ds); \$ds

\[\sqrt{2-2\,\cos x}\]\textgreater\$showev('integrate(ds,x,0,2*pi)) // hitung panjang sikloid satu putaran penuh

\[\int_{0}^{2\,\pi}{\sqrt{2-2\,\cos x}\;dx}=8\]\textgreater integrate(mxm(``ds''),0,2*pi) // hitung secara numerik

\begin{verbatim}
8
\end{verbatim}

\textgreater romberg(mxm(``ds''),0,2*pi) // cara lain hitung secara numerik

\begin{verbatim}
8
\end{verbatim}

\section{Kurvatur (Kelengkungan) Kurva}\label{kurvatur-kelengkungan-kurva}

image: Osculating.png

Aslinya, kelengkungan kurva diferensiabel (yakni, kurva mulus yang tidak lancip) di titik P didefinisikan melalui lingkaran oskulasi (yaitu, lingkaran yang melalui titik P dan terbaik memperkirakan, paling banyak menyinggung kurva di sekitar P). Pusat dan radius kelengkungan kurva di P adalah pusat dan radius lingkaran oskulasi. Kelengkungan adalah kebalikan dari radius kelengkungan:

\[\kappa =\frac {1}{R}\]dengan R adalah radius kelengkungan. (Setiap lingkaran memiliki kelengkungan ini pada setiap titiknya, dapat diartikan, setiap lingkaran berputar 2pi sejauh 2piR.)

Definisi ini sulit dimanipulasi dan dinyatakan ke dalam rumus untuk kurva umum. Oleh karena itu digunakan definisi lain yang ekivalen.

\section{Definisi Kurvatur dengan Fungsi Parametrik Panjang Kurva}\label{definisi-kurvatur-dengan-fungsi-parametrik-panjang-kurva}

Setiap kurva diferensiabel dapat dinyatakan dengan persamaan parametrik terhadap panjang kurva s:

\[\gamma(s) = (x(s),\ y(s)),\]dengan x dan y adalah fungsi riil yang diferensiabel, yang memenuhi:

\[\|\gamma'(s)\|=\sqrt{x'(s)^2+y'(s)^2}=1.\]Ini berarti bahwa vektor singgung

\[\mathbf{T}(s)=(x'(s),\ y'(s))\]memiliki norm 1 dan merupakan vektor singgung satuan.

Apabila kurvanya memiliki turunan kedua, artinya turunan kedua x dan y ada, maka T'(s) ada. Vektor ini merupakan normal kurva yang arahnya menuju pusat kurvatur, norm-nya merupakan nilai kurvatur (kelengkungan):

\[ \begin{aligned}\mathbf{T}(s) &= \mathbf{\gamma}'(s),\\ \mathbf{T}^{2}(s) &=1\ \text{(konstanta)}\Rightarrow \mathbf{T}'(s)\cdot \mathbf{T}(s)=0\\ \kappa(s) &=\|\mathbf {T}'(s)\|= \|\mathbf{\gamma}''(s)\|=\sqrt{x''(s)^{2}+y''(s)^{2}}.\end{aligned}\]Nilai

\[R(s)=\frac{1}{\kappa(s)}\]disebut jari-jari (radius) kelengkungan kurva.

Bilangan riil

\[ k(s) = \pm\kappa(s)\]disebut nilai kelengkungan bertanda.

Contoh:

Akan ditentukan kurvatur lingkaran

\[x=r\cos t,\ y= r\sin t.\]\textgreater fx \&= r*cos(t); fy \&=r*sin(t);

\textgreater\&assume(t\textgreater0,r\textgreater0); s \&=integrate(sqrt(diff(fx,t)\textsuperscript{2+diff(fy,t)}2),t,0,t); s // elemen panjang kurva, panjang busur lingkaran (s)

\begin{verbatim}
                                 r t
\end{verbatim}

\textgreater\&kill(s); fx \&= r*cos(s/r); fy \&=r*sin(s/r); // definisi ulang persamaan parametrik terhadap s dengan substitusi t=s/r

\textgreater k \&= trigsimp(sqrt(diff(fx,s,2)\textsuperscript{2+diff(fy,s,2)}2)); \$k // nilai kurvatur lingkaran dengan menggunakan definisi di atas

\[\frac{1}{r}\]Untuk representasi parametrik umum, misalkan

\[x = x(t),\ y= y(t)\]merupakan persamaan parametrik untuk kurva bidang yang terdiferensialkan dua kali. Kurvatur untuk kurva tersebut didefinisikan sebagai

\[\begin{aligned}\kappa &= \frac{d\phi}{ds}=\frac{\frac{d\phi}{dt}}{\frac{ds}{dt}}\quad (\phi \text{ adalah sudut kemiringan garis singgung dan }s \text{ adalah panjang kurva})\\ &=\frac{\frac{d\phi}{dt}}{\sqrt{(\frac{dx}{dt})^2+(\frac{dy}{dt})^2}}= \frac{\frac{d\phi}{dt}}{\sqrt{x'(t)^2+y'(t)^2}}.\end{aligned}.\]Selanjutnya, pembilang pada persamaan di atas dapat dicari sebagai berikut.

\[\begin{aligned}\sec^2\phi\frac{d\phi}{dt} &= \frac{d}{dt}\left(\tan\phi\right)= \frac{d}{dt}\left(\frac{dy}{dx}\right)= \frac{d}{dt}\left(\frac{dy/dt}{dx/dt}\right)= \frac{d}{dt}\left(\frac{y'(t)}{x'(t)}\right)=\frac{x'(t)y''(t)-x''(t)y'(t)}{x'(t)^2}.\\ & \\ \frac{d\phi}{dt} &= \frac{1}{\sec^2\phi}\frac{x'(t)y''(t)-x''(t)y'(t)}{x'(t)^2}\\ &= \frac{1}{1+\tan^2\phi}\frac{x'(t)y''(t)-x''(t)y'(t)}{x'(t)^2}\\ &= \frac{1}{1+\left(\frac{y'(t)}{x'(t)}\right)^2}\frac{x'(t)y''(t)-x''(t)y'(t)}{x'(t)^2}\\ &= \frac{x'(t)y''(t)-x''(t)y'(t)}{x'(t)^2+y'(t)^2}.\end{aligned}\]Jadi, rumus kurvatur untuk kurva parametrik

\[x=x(t),\ y=y(t)\]adalah

\[\kappa(t) = \frac{x'(t)y''(t)-x''(t)y'(t)}{\left(x'(t)^2+y'(t)^2\right)^{3/2}}.\]Jika kurvanya dinyatakan dengan persamaan parametrik pada koordinat kutub

\[x=r(\theta)\cos\theta,\ y=r(\theta)\sin\theta,\]maka rumus kurvaturnya adalah

\[\kappa(\theta) = \frac{r(\theta)^2+2r'(\theta)^2-r(\theta)r''(\theta)}{\left(r'(\theta)^2+r'(\theta)^2\right)^{3/2}}.\](Silakan Anda turunkan rumus tersebut!)

Contoh:

Lingkaran dengan pusat (0,0) dan jari-jari r dapat dinyatakan dengan persamaan parametrik

\[x=r\cos t,\ y=r\sin t.\]Nilai kelengkungan lingkaran tersebut adalah

\[\kappa(t)=\frac{x'(t)y''(t)-x''(t)y'(t)}{\left(x'(t)^2+y'(t)^2\right)^{3/2}}=\frac{r^2}{r^3}=\frac 1 r.\]Hasil cocok dengan definisi kurvatur suatu kelengkungan.

Kurva

\[y=f(x)\]dapat dinyatakan ke dalam persamaan parametrik

\[x=t,\ y=f(t),\ \text{ dengan } x'(t)=1,\ x''(t)=0,\]sehingga kurvaturnya adalah

\[\kappa(t) = \frac{y''(t)}{\left(1+y'(t)^2\right)^{3/2}}.\]Contoh:

Akan ditentukan kurvatur parabola

\[y=ax^2+bx+c.\]\textgreater function f(x) \&= a*x\^{}2+b*x+c; \$y=f(x)

\[y=a\,x^2+b\,x+c\]\textgreater function k(x) \&= (diff(f(x),x,2))/(1+diff(f(x),x)\textsuperscript{2)}(3/2); \$'k(x)=k(x) // kelengkungan parabola

\[k\left(x\right)=\frac{2\,a}{\left(\left(2\,a\,x+b\right)^2+1\right)  ^{\frac{3}{2}}}\]\textgreater function f(x) \&= x\^{}2+x+1; \$y=f(x) // akan kita plot kelengkungan parabola untuk a=b=c=1

\[y=x^2+x+1\]\textgreater function k(x) \&= (diff(f(x),x,2))/(1+diff(f(x),x)\textsuperscript{2)}(3/2); \$'k(x)=k(x) // kelengkungan parabola

\[k\left(x\right)=\frac{2}{\left(\left(2\,x+1\right)^2+1\right)^{  \frac{3}{2}}}\]Berikut kita gambar parabola tersebut beserta kurva kelengkungan, kurva jari-jari kelengkungan dan salah satu lingkaran oskulasi di titik puncak parabola. Perhatikan, puncak parabola dan jari-jari lingkaran oskulasi di puncak parabola adalah

\[(-1/2,3/4),\ 1/k(2)=1/2,\]sehingga pusat lingkaran oskulasi adalah (-1/2, 5/4).

\textgreater plot2d({[}``f(x)'', ``k(x)''{]},-2,1, color={[}blue,red{]}); plot2d(``1/k(x)'',-1.5,1,color=green,\textgreater add); \ldots{}\\
\textgreater{} plot2d(``-1/2+1/k(-1/2)*cos(x)'',``5/4+1/k(-1/2)*sin(x)'',xmin=0,xmax=2pi,\textgreater add,color=blue):

\begin{figure}
\centering
\pandocbounded{\includegraphics[keepaspectratio]{images/KALKULUS_Diva Nagita(23030630024)-312.png}}
\caption{images/KALKULUS\_Diva\%20Nagita(23030630024)-312.png}
\end{figure}

Untuk kurva yang dinyatakan dengan fungsi implisit

\[F(x,y)=0\]dengan turunan-turunan parsial

\[F_x=\frac{\partial F}{\partial x},\ F_y=\frac{\partial F}{\partial y},\ F_{xy}=\frac{\partial}{\partial y}\left(\frac{\partial F}{\partial x}\right),\ F_{xx}=\frac{\partial}{\partial x}\left(\frac{\partial F}{\partial x}\right),\ F_{yy}=\frac{\partial}{\partial y}\left(\frac{\partial F}{\partial y}\right),\]berlaku

\[F_x dx+ F_y dy = 0\text{ atau } \frac{dy}{dx}=-\frac{F_x}{F_y},\]sehingga kurvaturnya adalah

\[\kappa =\frac {F_y^2F_{xx}-2F_xF_yF_{xy}+F_x^2F_{yy}}{\left(F_x^2+F_y^2\right)^{3/2}}.\](Silakan Anda turunkan sendiri!)

Contoh 1:

Parabola

\[y=ax^2+bx+c\]dapat dinyatakan ke dalam persamaan implisit

\[ax^2+bx+c-y=0.\]\textgreater function F(x,y) \&=a*x\^{}2+b*x+c-y; \$F(x,y)

\[-y+a\,x^2+b\,x+c\]\textgreater Fx \&= diff(F(x,y),x), Fxx \&=diff(F(x,y),x,2), Fy \&=diff(F(x,y),y), Fxy \&=diff(diff(F(x,y),x),y), Fyy \&=diff(F(x,y),y,2)

\begin{verbatim}
                              2 a x + b


                                 2 a


                                 - 1


                                  0


                                  0
\end{verbatim}

\textgreater function k(x) \&= (Fy\textsuperscript{2*Fxx-2*Fx*Fy*Fxy+Fx}2*Fyy)/(Fx\textsuperscript{2+Fy}2)\^{}(3/2); \$'k(x)=k(x) // kurvatur parabola tersebut

\[k\left(x\right)=\frac{2\,a}{\left(\left(2\,a\,x+b\right)^2+1\right)  ^{\frac{3}{2}}}\]\# Barisan dan Deret Barisan adalah susunan bilangan yang memiliki pola

atau karakteristik tertentu, sedangkan deret adalah hasil penjumlahan dari anggota-anggota dalam barian tertentu.

Barisan dapat didefinisikan dengan beberapa cara di dalam EMT, di antaranya: + dengan cara yang sama seperti mendefinisikan vektor dengan elemen-elemen beraturan (menggunakan titik dua ``:''); + menggunakan perintah ``sequence'' dan rumus barisan (suku ke -n); + menggunakan perintah ``iterate'' atau ``niterate''; + menggunakan fungsi Maxima ``create\_list'' atau ``makelist'' untuk - menghasilkan barisan simbolik; - menggunakan fungsi biasa yang inputnya vektor atau barisan; - menggunakan fungsi rekursif.

EMT menyediakan beberapa perintah (fungsi) terkait barisan, yakni:

\begin{itemize}
\item
  sum: menghitung jumlah semua elemen suatu barisan
\item
  cumsum: jumlah kumulatif suatu barisan
\item
  differences: selisih antar elemen-elemen berturutan
\end{itemize}

Contoh:

\textgreater5:15

\begin{verbatim}
[5,  6,  7,  8,  9,  10,  11,  12,  13,  14,  15]
\end{verbatim}

\textgreater1:3:10

\begin{verbatim}
[1,  4,  7,  10]
\end{verbatim}

\textgreater sum(1:3:10)

\begin{verbatim}
22
\end{verbatim}

\section{Rumus deret berhingga}\label{rumus-deret-berhingga}

Diberikan deret sederhana sebagai berikut

\textgreater A := {[}2, 4, 8, 16, 32{]}

\begin{verbatim}
[2,  4,  8,  16,  32]
\end{verbatim}

Menentukan suku ke-n

\[a_n=ar^{n-1}\]Contoh:

Tentukanlah suku ke-7 dari deret tersebut

Dengan a=2 dan r=2

\textgreater n=7; a=2; r=2; a*r\^{}(n-1)

\begin{verbatim}
128
\end{verbatim}

Menghitung Jumlah deret suku ke-n

\[S_n=\frac{a(r^n-1)}{r-1}\]

jika konvergen, r \textgreater{} 1

atau

\[S_n=\frac{a(1-r^n)}{1-r}\]

jika divergen, r \textless{} 1

Contoh:

Hitunglah jumlah deret A pada suku ke-5

Dengan a=2 dan r=2

\textgreater n=5; a=2; r=2; a*((r\^{}n)-1)/(r-1)

\begin{verbatim}
62
\end{verbatim}

\textgreater sum(A := {[}2, 4, 8, 16, 32{]})

\begin{verbatim}
62
\end{verbatim}

\section{Rumus Deret Tak Hingga}\label{rumus-deret-tak-hingga}

Menghitung jumlah suku tak hingga konvergen

Untuk -1 \textless{} r \textless{} 1, maka

\[S_\infty=U_1+U_2+U_3+...\]

Memiliki limit jumlah

\[S_\infty=\frac{a}{1-r}\]

dengan

\[U_1=a\]Contoh:

\[S_\infty=12+6+3+...\]\textgreater a=12; r=1/2; a/(1-r)

\begin{verbatim}
24
\end{verbatim}

\section{Limit Barisan}\label{limit-barisan}

Limit barisan melambangkan nilai mutlak untuk bilangan riil dan nilai modulus untuk bilangan kompleks.

Definisi formal:

\[\forall \varepsilon > 0, \exists k \in \mathbb{N} : (\forall n \in \mathbb{N}, n \geq k \implies |x_n - L| <\varepsilon)\]Dapat dinotasikan sebagai

\[\lim_{n \to \infty}x_n=L\]Contoh:

\begin{enumerate}
\def\labelenumi{\arabic{enumi}.}
\tightlist
\item
\end{enumerate}

\[\lim_{n \to \infty}\frac{1}{n}\]\textgreater\$showev('limit(1/n,n,inf))

\[\lim_{n\rightarrow \infty }{\frac{1}{n}}=0\]2.

\[\lim_{n\to\infty}\frac{2n^2-1}{n^2+5}\]\textgreater\$showev('limit((2*n\textsuperscript{2-1)/(n}2+5),n,inf))

\[\lim_{n\rightarrow \infty }{\frac{2\,n^2-1}{n^2+5}}=2\]\# Iterasi dan Barisan

EMT menyediakan fungsi iterate(``g(x)'', x0, n) untuk melakukan iterasi

\[x_{k+1}=g(x_k), \ x_0=x_0, k= 1, 2, 3, ..., n.\]Berikut ini disajikan contoh-contoh penggunaan iterasi dan rekursi dengan EMT. Contoh pertama menunjukkan pertumbuhan dari nilai awal 1000 dengan laju pertambahan 5\%, selama 10 periode.

\textgreater q=1.05; iterate(``x*q'',1000,n=10)'

\begin{verbatim}
         1000 
         1050 
       1102.5 
      1157.63 
      1215.51 
      1276.28 
       1340.1 
       1407.1 
      1477.46 
      1551.33 
      1628.89 
\end{verbatim}

Contoh berikutnya memperlihatkan bahaya menabung di bank pada masa sekarang! Dengan bunga tabungan sebesar 6\% per tahun atau 0.5\% per bulan dipotong pajak 20\%, dan biaya administrasi 10000 per bulan, tabungan sebesar 1 juta tanpa diambil selama sekitar 10 tahunan akan habis diambil oleh bank!

\textgreater r=0.005; plot2d(iterate(``(1+0.8*r)*x-10000'',1000000,n=130)):

\begin{figure}
\centering
\pandocbounded{\includegraphics[keepaspectratio]{images/KALKULUS_Diva Nagita(23030630024)-335.png}}
\caption{images/KALKULUS\_Diva\%20Nagita(23030630024)-335.png}
\end{figure}

Silakan Anda coba-coba, dengan tabungan minimal berapa agar tidak akan habis diambil oleh bank dengan ketentuan bunga dan biaya administrasi seperti di atas.

Berikut adalah perhitungan minimal tabungan agar aman di bank dengan bunga sebesar r dan biaya administrasi a, pajak bunga 20\%.

\textgreater\$solve(0.8*r*A-a,A), \$\% with {[}r=0.005, a=10{]}

\[\left[ A=2500.0 \right] \]\pandocbounded{\includegraphics[keepaspectratio]{images/KALKULUS_Diva Nagita(23030630024)-337.png}}

Berikut didefinisikan fungsi untuk menghitung saldo tabungan, kemudian dilakukan iterasi.

\textgreater function saldo(x,r,a) := round((1+0.8*r)*x-a,2);

\textgreater iterate(\{\{``saldo'',0.005,10\}\},1000,n=6)

\begin{verbatim}
[1000,  994,  987.98,  981.93,  975.86,  969.76,  963.64]
\end{verbatim}

\textgreater iterate(\{\{``saldo'',0.005,10\}\},2000,n=6)

\begin{verbatim}
[2000,  1998,  1995.99,  1993.97,  1991.95,  1989.92,  1987.88]
\end{verbatim}

\textgreater iterate(\{\{``saldo'',0.005,10\}\},2500,n=6)

\begin{verbatim}
[2500,  2500,  2500,  2500,  2500,  2500,  2500]
\end{verbatim}

Tabungan senilai 2,5 juta akan aman dan tidak akan berubah nilai (jika tidak ada penarikan), sedangkan jika tabungan awal kurang dari 2,5 juta, lama kelamaan akan berkurang meskipun tidak pernah dilakukan penarikan uang tabungan.

\textgreater iterate(\{\{``saldo'',0.005,10\}\},3000,n=6)

\begin{verbatim}
[3000,  3002,  3004.01,  3006.03,  3008.05,  3010.08,  3012.12]
\end{verbatim}

Tabungan yang lebih dari 2,5 juta baru akan bertambah jika tidak ada penarikan.

Untuk barisan yang lebih kompleks dapat digunakan fungsi ``sequence()''. Fungsi ini menghitung nilai-nilai x{[}n{]} dari semua nilai sebelumnya, x{[}1{]},\ldots,x{[}n-1{]} yang diketahui.

Berikut adalah contoh barisan Fibonacci.

\[x_n = x_{n-1}+x_{n-2}, \quad x_1=1, \quad x_2 =1\]\textgreater sequence(``x{[}n-1{]}+x{[}n-2{]}'',{[}1,1{]},15)

\begin{verbatim}
[1,  1,  2,  3,  5,  8,  13,  21,  34,  55,  89,  144,  233,  377,  610]
\end{verbatim}

Barisan Fibonacci memiliki banyak sifat menarik, salah satunya adalah akar pangkat ke-n suku ke-n akan konvergen ke pecahan emas:

\textgreater\$'(1+sqrt(5))/2=float((1+sqrt(5))/2)

\[\frac{\sqrt{5}+1}{2}=1.618033988749895\]\textgreater plot2d(sequence(``x{[}n-1{]}+x{[}n-2{]}'',{[}1,1{]},250)\^{}(1/(1:250))):

\begin{figure}
\centering
\pandocbounded{\includegraphics[keepaspectratio]{images/KALKULUS_Diva Nagita(23030630024)-340.png}}
\caption{images/KALKULUS\_Diva\%20Nagita(23030630024)-340.png}
\end{figure}

Barisan yang sama juga dapat dihasilkan dengan menggunakan loop.

\textgreater x=ones(500); for k=3 to 500; x{[}k{]}=x{[}k-1{]}+x{[}k-2{]}; end;

Rekursi dapat dilakukan dengan menggunakan rumus yang tergantung pada semua elemen sebelumnya. Pada contoh berikut, elemen ke-n merupakan jumlah (n-1) elemen sebelumnya, dimulai dengan 1 (elemen ke-1). Jelas, nilai elemen ke-n adalah 2\^{}(n-2), untuk n=2, 4, 5, \ldots.

\textgreater sequence(``sum(x)'',1,10)

\begin{verbatim}
[1,  1,  2,  4,  8,  16,  32,  64,  128,  256]
\end{verbatim}

Selain menggunakan ekspresi dalam x dan n, kita juga dapat menggunakan fungsi.

Pada contoh berikut, digunakan iterasi

\[x_n =A \cdot x_{n-1},\]dengan A suatu matriks 2x2, dan setiap x{[}n{]} merupakan matriks/vektor 2x1.

\textgreater A={[}1,1;1,2{]}; function suku(x,n) := A.x{[},n-1{]}

\textgreater sequence(``suku'',{[}1;1{]},6)

\begin{verbatim}
Real 2 x 6 matrix

            1             2             5            13     ...
            1             3             8            21     ...
\end{verbatim}

Hasil yang sama juga dapat diperoleh dengan menggunakan fungsi perpangkatan matriks ``matrixpower()''. Cara ini lebih cepat, karena hanya menggunakan perkalian matriks sebanyak log\_2(n).

\[x_n=A.x_{n-1}=A^2.x_{n-2}=A^3.x_{n-3}= ... = A^{n-1}.x_1.\]\textgreater sequence(``matrixpower(A,n).{[}1;1{]}'',1,6)

\begin{verbatim}
Real 2 x 6 matrix

            1             5            13            34     ...
            1             8            21            55     ...
\end{verbatim}

\chapter{Spiral Theodorus}\label{spiral-theodorus}

image: Spiral\_of\_Theodorus.png

Spiral Theodorus (spiral segitiga siku-siku) dapat digambar secara rekursif. Rumus rekursifnya adalah:

\[x_n = \left( 1 + \frac{i}{\sqrt{n-1}} \right) \, x_{n-1}, \quad x_1=1,\]yang menghasilkan barisan bilangan kompleks.

\textgreater function g(n) := 1+I/sqrt(n)

Rekursinya dapat dijalankan sebanyak 17 untuk menghasilkan barisan 17 bilangan kompleks, kemudian digambar bilangan-bilangan kompleksnya.

\textgreater x=sequence(``g(n-1)*x{[}n-1{]}'',1,17); plot2d(x,r=3.5); textbox(latex(``Spiral\textbackslash{} Theodorus''),0.4):

\begin{figure}
\centering
\pandocbounded{\includegraphics[keepaspectratio]{images/KALKULUS_Diva Nagita(23030630024)-344.png}}
\caption{images/KALKULUS\_Diva\%20Nagita(23030630024)-344.png}
\end{figure}

Selanjutnya dihubungan titik 0 dengan titik-titik kompleks tersebut menggunakan loop.

\textgreater for i=1:cols(x); plot2d({[}0,x{[}i{]}{]},\textgreater add); end:

\begin{figure}
\centering
\pandocbounded{\includegraphics[keepaspectratio]{images/KALKULUS_Diva Nagita(23030630024)-345.png}}
\caption{images/KALKULUS\_Diva\%20Nagita(23030630024)-345.png}
\end{figure}

\textgreater{}

Spiral tersebut juga dapat didefinisikan menggunakan fungsi rekursif, yang tidak memmerlukan indeks dan bilangan kompleks. Dalam hal ini diigunakan vektor kolom pada bidang.

\textgreater function gstep (v) \ldots{}

\begin{verbatim}
w=[-v[2];v[1]];
return v+w/norm(w);
endfunction
\end{verbatim}

Jika dilakukan iterasi 16 kali dimulai dari {[}1;0{]} akan didapatkan matriks yang memuat vektor-vektor dari setiap iterasi.

\textgreater x=iterate(``gstep'',{[}1;0{]},16); plot2d(x{[}1{]},x{[}2{]},r=3.5,\textgreater points):

\begin{figure}
\centering
\pandocbounded{\includegraphics[keepaspectratio]{images/KALKULUS_Diva Nagita(23030630024)-346.png}}
\caption{images/KALKULUS\_Diva\%20Nagita(23030630024)-346.png}
\end{figure}

\section{Kekonvergenan}\label{kekonvergenan}

Barisan yang konvergen adalah barisan yang terbatas.

Jika (Xn) konvergen, maka (Xn) terbatas.

Kita dapat menggunakan fungsi iterate (``g(x)'', x0, n) untuk melakukan iterasi

\[x_{k+1}=g(x_k), \ x_0=x_0, k= 1, 2, 3, ..., n.\]Contoh:

\textgreater iterate(``cos(x)'',1) // iterasi x(n+1)=cos(x(n)), dengan x(0)=1.

\begin{verbatim}
0.739085133216
\end{verbatim}

Iterasi tersebut konvergen ke penyelesaian persamaan

\[x = \cos(x).\]Iterasi ini juga dapat dilakukan pada interval, hasilnya adalah barisan interval yang memuat akar tersebut.

\textgreater hasil := iterate(``cos(x)'',\textsubscript{2,3}) //interval awal (2,3)

\begin{verbatim}
~0.7390851332143,0.7390851332161~
\end{verbatim}

Jika interval hasil tersebut sedikit diperlebar, akan terlihat bahwa interval tersebut memuat akar persamaan x=cos(x).

\textgreater h=expand(hasil,100), cos(h) \textless\textless{} h

\begin{verbatim}
~0.73908513313,0.7390851333~
1
\end{verbatim}

Iterasi juga dapat digunakan pada fungsi yang didefinisikan.

\textgreater function f(x) := (x+4/x)/4

Iterasi x(n+1)=f(x(n)) akan konvergen ke akar kuadrat 4.

\textgreater iterate(``f'',4), sqrt(4)

\begin{verbatim}
1.15470053838
2
\end{verbatim}

\textgreater iterate(``f'',2,5)

\begin{verbatim}
[2,  1,  1.25,  1.1125,  1.177,  1.14387]
\end{verbatim}

Untuk iterasi ini tidak dapat dilakukan terhadap interval.

\textgreater niterate(``f'',\textsubscript{1,2},5)

\begin{verbatim}
[ ~1,2~,  ~0.75,1.5~,  ~0.85,1.8~,  ~0.79,1.6~,  ~0.82,1.7~,  ~0.81,1.7~ ]
\end{verbatim}

Perhatikan, hasil iterasinya sama dengan interval awal. Alasannya adalah perhitungan dengan interval bersifat terlalu longgar. Untuk meingkatkan perhitungan pada ekspresi dapat digunakan pembagian intervalnya, menggunakan fungsi ieval().

\textgreater function s(x) := ieval(``(x+2/x)/2'',x,10)

Selanjutnya dapat dilakukan iterasi hingga diperoleh hasil optimal, dan intervalnya tidak semakin mengecil. Hasilnya berupa interval yang memuat akar persamaan:

\[x = \frac{1}{2} \left( x + \frac{2}{x} \right).\]Satu-satunya solusi adalah

\[x = \sqrt2.\]\textgreater iterate(``s'',\textsubscript{1,2})

\begin{verbatim}
~1.41421356236,1.41421356239~
\end{verbatim}

Fungsi ``iterate()'' juga dapat bekerja pada vektor. Berikut adalah contoh fungsi vektor, yang menghasilkan rata-rata aritmetika dan rata-rata geometri.

\[(a_{n+1},b_{n+1}) = \left( \frac{a_n+b_n}{2}, \sqrt{a_nb_n} \right)\]Iterasi ke-n disimpan pada vektor kolom x{[}n{]}.

\textgreater function g(x) := {[}(x{[}1{]}+x{[}2{]})/2;sqrt(x{[}1{]}*x{[}2{]}){]}

Iterasi dengan menggunakan fungsi tersebut akan konvergen ke rata-rata aritmetika dan geometri dari nilai-nilai awal.

\textgreater iterate(``g'',{[}1;5{]})

\begin{verbatim}
      2.60401 
      2.60401 
\end{verbatim}

Hasil tersebut konvergen agak cepat, seperti kita cek sebagai berikut.

\textgreater iterate(``g'',{[}1;5{]},4)

\begin{verbatim}
            1             3       2.61803       2.60403       2.60401 
            5       2.23607       2.59002       2.60399       2.60401 
\end{verbatim}

Iterasi pada interval dapat dilakukan dan stabil, namun tidak menunjukkan bahwa limitnya pada batas-batas yang dihitung.

\textgreater iterate(``g'',{[}\textsubscript{1};\textsubscript{5}{]},4)

\begin{verbatim}
Interval 2 x 5 matrix

~0.999999999999999778,1.00000000000000022~     ...
~4.99999999999999911,5.00000000000000089~     ...
\end{verbatim}

Iterasi berikut konvergen sangat lambat.

\[x_{n+1} = \sqrt{x_n}.\]\textgreater iterate(``sqrt(x)'',2,10)

\begin{verbatim}
[2,  1.41421,  1.18921,  1.09051,  1.04427,  1.0219,  1.01089,
1.00543,  1.00271,  1.00135,  1.00068]
\end{verbatim}

Kekonvergenan iterasi tersebut dapat dipercepatdengan percepatan Steffenson:

\textgreater steffenson(``sqrt(x)'',2,10)

\begin{verbatim}
[1.04888,  1.00028,  1,  1]
\end{verbatim}

\chapter{Iterasi menggunakan Loop yang ditulis Langsung}\label{iterasi-menggunakan-loop-yang-ditulis-langsung}

Berikut adalah beberapa contoh penggunaan loop untuk melakukan iterasi yang ditulis langsung pada baris perintah.

\textgreater x=2; repeat x=(x+2/x)/2; until x\^{}2\textasciitilde=2; end; x,

\begin{verbatim}
1.41421356237
\end{verbatim}

Penggabungan matriks menggunakan tanda ``\textbar{}'' dapat digunakan untuk menyimpan semua hasil iterasi.

\textgreater v={[}1{]}; for i=2 to 8; v=v\textbar(v{[}i-1{]}*i); end; v,

\begin{verbatim}
[1,  2,  6,  24,  120,  720,  5040,  40320]
\end{verbatim}

hasil iterasi juga dapat disimpan pada vektor yang sudah ada.

\textgreater v=ones(1,100); for i=2 to cols(v); v{[}i{]}=v{[}i-1{]}*i; end; \ldots{}\\
\textgreater{} plot2d(v,logplot=1); textbox(latex(\&log(n)),x=0.5):

\begin{figure}
\centering
\pandocbounded{\includegraphics[keepaspectratio]{images/KALKULUS_Diva Nagita(23030630024)-353.png}}
\caption{images/KALKULUS\_Diva\%20Nagita(23030630024)-353.png}
\end{figure}

\textgreater A ={[}0.5,0.2;0.7,0.1{]}; b={[}2;2{]}; \ldots{}\\
\textgreater{} x={[}1;1{]}; repeat xnew=A.x-b; until all(xnew\textasciitilde=x); x=xnew; end; \ldots{}\\
\textgreater{} x,

\begin{verbatim}
     -7.09677 
     -7.74194 
\end{verbatim}

\chapter{Iterasi di dalam Fungsi}\label{iterasi-di-dalam-fungsi}

Fungsi atau program juga dapat menggunakan iterasi dan dapat digunakan untuk melakukan iterasi. Berikut adalah beberapa contoh iterasi di dalam fungsi.

Contoh berikut adalah suatu fungsi untuk menghitung berapa lama suatu iterasi konvergen. Nilai fungsi tersebut adalah hasil akhir iterasi dan banyak iterasi sampai konvergen.

\textgreater function map hiter(f\$,x0) \ldots{}

\begin{verbatim}
x=x0;
maxiter=0;
repeat
  xnew=f$(x);
  maxiter=maxiter+1;
  until xnew~=x;
  x=xnew;
end;
return maxiter;
endfunction
\end{verbatim}

Misalnya, berikut adalah iterasi untuk mendapatkan hampiran akar kuadrat 2, cukup cepat, konvergen pada iterasi ke-5, jika dimulai dari hampiran awal 2.

\textgreater hiter(``(x+2/x)/2'',2)

\begin{verbatim}
5
\end{verbatim}

Karena fungsinya didefinisikan menggunakan ``map''. maka nilai awalnya dapat berupa vektor.

\textgreater x=1.5:0.1:10; hasil=hiter(``(x+2/x)/2'',x); \ldots{}\\
\textgreater{} plot2d(x,hasil):

\begin{figure}
\centering
\pandocbounded{\includegraphics[keepaspectratio]{images/KALKULUS_Diva Nagita(23030630024)-354.png}}
\caption{images/KALKULUS\_Diva\%20Nagita(23030630024)-354.png}
\end{figure}

Dari gambar di atas terlihat bahwa kekonvergenan iterasinya semakin lambat, untuk nilai awal semakin besar, namun penambahnnya tidak kontinu. Kita dapat menemukan kapan maksimum iterasinya bertambah.

\textgreater hasil{[}1:10{]}

\begin{verbatim}
[4,  5,  5,  5,  5,  5,  6,  6,  6,  6]
\end{verbatim}

\textgreater x{[}nonzeros(differences(hasil)){]}

\begin{verbatim}
[1.5,  2,  3.4,  6.6]
\end{verbatim}

maksimum iterasi sampai konvergen meningkat pada saat nilai awalnya 1.5, 2, 3.4, dan 6.6.

Contoh berikutnya adalah metode Newton pada polinomial kompleks berderajat 3.

\textgreater p \&= x\^{}3-1; newton \&= x-p/diff(p,x); \$newton

\[x-\frac{x^3-1}{3\,x^2}\]Selanjutnya didefinisikan fungsi untuk melakukan iterasi (aslinya 10 kali).

\textgreater function iterasi(f\$,x,n=10) \ldots{}

\begin{verbatim}
loop 1 to n; x=f$(x); end;
return x;
endfunction
\end{verbatim}

Kita mulai dengan menentukan titik-titik grid pada bidang kompleksnya.

\textgreater r=1.5; x=linspace(-r,r,501); Z=x+I*x'; W=iterasi(newton,Z);

Berikut adalah akar-akar polinomial di atas.

\textgreater z=\&solve(p)()

\begin{verbatim}
[ -0.5+0.866025i,  -0.5-0.866025i,  1+0i  ]
\end{verbatim}

Untuk menggambar hasil iterasinya, dihitung jarak dari hasil iterasi ke-10 ke masing-masing akar, kemudian digunakan untuk menghitung warna yang akan digambar, yang menunjukkan limit untuk masing-masing nilai awal.

Fungsi plotrgb() menggunakan jendela gambar terkini untuk menggambar warna RGB sebagai matriks.

\textgreater C=rgb(max(abs(W-z{[}1{]}),1),max(abs(W-z{[}2{]}),1),max(abs(W-z{[}3{]}),1)); \ldots{}\\
\textgreater{} plot2d(none,-r,r,-r,r); plotrgb(C):

\begin{figure}
\centering
\pandocbounded{\includegraphics[keepaspectratio]{images/KALKULUS_Diva Nagita(23030630024)-356.png}}
\caption{images/KALKULUS\_Diva\%20Nagita(23030630024)-356.png}
\end{figure}

\chapter{Iterasi Simbolik}\label{iterasi-simbolik}

Seperti sudah dibahas sebelumnya, untuk menghasilkan barisan ekspresi simbolik dengan Maxima dapat digunakan fungsi makelist().

\textgreater\&powerdisp:true // untuk menampilkan deret pangkat mulai dari suku berpangkat terkecil

\begin{verbatim}
                                 true
\end{verbatim}

\textgreater deret \&= makelist(taylor(exp(x),x,0,k),k,1,3); \$deret // barisan deret Taylor untuk e\^{}x

\[\left[ 1+x , 1+x+\frac{x^2}{2} , 1+x+\frac{x^2}{2}+\frac{x^3}{6}   \right] \]Untuk mengubah barisan deret tersebut menjadi vektor string di EMT digunakan fungsi mxm2str(). Selanjutnya, vektor string/ekspresi hasilnya dapat digambar seperti menggambar vektor eskpresi pada EMT.

\textgreater plot2d(``exp(x)'',0,3); // plot fungsi aslinya, e\^{}x

\textgreater plot2d(mxm2str(``deret''),\textgreater add,color=4:6): // plot ketiga deret taylor hampiran fungsi tersebut

\begin{figure}
\centering
\pandocbounded{\includegraphics[keepaspectratio]{images/KALKULUS_Diva Nagita(23030630024)-358.png}}
\caption{images/KALKULUS\_Diva\%20Nagita(23030630024)-358.png}
\end{figure}

Selain cara di atas dapat juga dengan cara menggunakan indeks pada vektor/list yang dihasilkan.

\textgreater\$deret{[}3{]}

\[1+x+\frac{x^2}{2}+\frac{x^3}{6}\]\textgreater plot2d({[}``exp(x)'',\&deret{[}1{]},\&deret{[}2{]},\&deret{[}3{]}{]},0,3,color=1:4):

\begin{figure}
\centering
\pandocbounded{\includegraphics[keepaspectratio]{images/KALKULUS_Diva Nagita(23030630024)-360.png}}
\caption{images/KALKULUS\_Diva\%20Nagita(23030630024)-360.png}
\end{figure}

\textgreater\$sum(sin(k*x)/k,k,1,5)

\[\sin x+\frac{\sin \left(2\,x\right)}{2}+\frac{\sin \left(3\,x  \right)}{3}+\frac{\sin \left(4\,x\right)}{4}+\frac{\sin \left(5\,x  \right)}{5}\]Berikut adalah cara menggambar kurva

\[y=\sin(x) + \dfrac{\sin 3x}{3} + \dfrac{\sin 5x}{5} + \ldots.\]\textgreater plot2d(\&sum(sin((2*k+1)*x)/(2*k+1),k,0,20),0,2pi):

\begin{figure}
\centering
\pandocbounded{\includegraphics[keepaspectratio]{images/KALKULUS_Diva Nagita(23030630024)-363.png}}
\caption{images/KALKULUS\_Diva\%20Nagita(23030630024)-363.png}
\end{figure}

Hal serupa juga dapat dilakukan dengan menggunakan matriks, misalkan kita akan menggambar kurva

\[y = \sum_{k=1}^{100} \dfrac{\sin(kx)}{k},\quad 0\le x\le 2\pi.\]\textgreater x=linspace(0,2pi,1000); k=1:100; y=sum(sin(k*x')/k)'; plot2d(x,y):

\chapter{Tabel Fungsi}\label{tabel-fungsi}

Terdapat cara menarik untuk menghasilkan barisan dengan ekspresi Maxima. Perintah mxmtable() berguna untuk menampilkan dan menggambar barisan dan menghasilkan barisan sebagai vektor kolom.

Sebagai contoh berikut adalah barisan turunan ke-n x\^{}x di x=1.

\textgreater mxmtable(``diffat(x\^{}x,x=1,n)'',``n'',1,8,frac=1);

\begin{verbatim}
        1 
        2 
        3 
        8 
       10 
       54 
      -42 
      944 
\end{verbatim}

\begin{figure}
\centering
\pandocbounded{\includegraphics[keepaspectratio]{images/KALKULUS_Diva Nagita(23030630024)-366.png}}
\caption{images/KALKULUS\_Diva\%20Nagita(23030630024)-366.png}
\end{figure}

\textgreater\$'sum(k, k, 1, n) = factor(ev(sum(k, k, 1, n),simpsum=true)) // simpsum:menghitung deret secara simbolik

\[\sum_{k=1}^{n}{k}=\frac{n\,\left(1+n\right)}{2}\]\textgreater\$'sum(1/(3\^{}k+k), k, 0, inf) = factor(ev(sum(1/(3\^{}k+k), k, 0, inf),simpsum=true))

\[\sum_{k=0}^{\infty }{\frac{1}{k+3^{k}}}=\sum_{k=0}^{\infty }{\frac{  1}{k+3^{k}}}\]Di sini masih gagal, hasilnya tidak dihitung.

\textgreater\$'sum(1/x\^{}2, x, 1, inf)= ev(sum(1/x\^{}2, x, 1, inf),simpsum=true) // ev: menghitung nilai ekspresi

\[\sum_{x=1}^{\infty }{\frac{1}{x^2}}=\frac{\pi^2}{6}\]\textgreater\$'sum((-1)\^{}(k-1)/k, k, 1, inf) = factor(ev(sum((-1)\^{}(x-1)/x, x, 1, inf),simpsum=true))

\[\sum_{k=1}^{\infty }{\frac{\left(-1\right)^{-1+k}}{k}}=-\sum_{x=1  }^{\infty }{\frac{\left(-1\right)^{x}}{x}}\]Di sini masih gagal, hasilnya tidak dihitung.

\textgreater\$'sum((-1)\^{}k/(2*k-1), k, 1, inf) = factor(ev(sum((-1)\^{}k/(2*k-1), k, 1, inf),simpsum=true))

\[\sum_{k=1}^{\infty }{\frac{\left(-1\right)^{k}}{-1+2\,k}}=\sum_{k=1  }^{\infty }{\frac{\left(-1\right)^{k}}{-1+2\,k}}\]\textgreater\$ev(sum(1/n!, n, 0, inf),simpsum=true)

\[\sum_{n=0}^{\infty }{\frac{1}{n!}}\]Di sini masih gagal, hasilnya tidak dihitung, harusnya hasilnya e.

\textgreater\&assume(abs(x)\textless1); \$'sum(a*x\^{}k, k, 0, inf)=ev(sum(a*x\^{}k, k, 0, inf),simpsum=true), \&forget(abs(x)\textless1);

\[a\,\sum_{k=0}^{\infty }{x^{k}}=\frac{a}{1-x}\]Deret geometri tak hingga, dengan asumsi rasional antara -1 dan 1.

\textgreater\$'sum(x\textsuperscript{k/k!,k,0,inf)=ev(sum(x}k/k!,k,0,inf),simpsum=true)

\[\sum_{k=0}^{\infty }{\frac{x^{k}}{k!}}=\sum_{k=0}^{\infty }{\frac{x  ^{k}}{k!}}\]\textgreater\$limit(sum(x\^{}k/k!,k,0,n),n,inf)

\[\lim_{n\rightarrow \infty }{\sum_{k=0}^{n}{\frac{x^{k}}{k!}}}\]\textgreater function d(n) \&= sum(1/(k\^{}2-k),k,2,n); \$'d(n)=d(n)

\[d\left(n\right)=\sum_{k=2}^{n}{\frac{1}{-k+k^2}}\]\textgreater\$d(10)=ev(d(10),simpsum=true)

\[\sum_{k=2}^{10}{\frac{1}{-k+k^2}}=\frac{9}{10}\]\textgreater\$d(100)=ev(d(100),simpsum=true)

\[\sum_{k=2}^{100}{\frac{1}{-k+k^2}}=\frac{99}{100}\]\# Deret Taylor

Deret Taylor suatu fungsi f yang diferensiabel sampai tak hingga di sekitar x=a adalah:

\[f(x) = \sum_{k=0}^\infty \frac{(x-a)^k f^{(k)}(a)}{k!}.\]\textgreater\$'e\^{}x =taylor(exp(x),x,0,10) // deret Taylor e\^{}x di sekitar x=0, sampai suku ke-11

\[e^{x}=1+x+\frac{x^2}{2}+\frac{x^3}{6}+\frac{x^4}{24}+\frac{x^5}{120  }+\frac{x^6}{720}+\frac{x^7}{5040}+\frac{x^8}{40320}+\frac{x^9}{  362880}+\frac{x^{10}}{3628800}\]\textgreater\$'log(x)=taylor(log(x),x,1,10)// deret log(x) di sekitar x=1

\[\log x=-1-\frac{\left(-1+x\right)^2}{2}+\frac{\left(-1+x\right)^3}{  3}-\frac{\left(-1+x\right)^4}{4}+\frac{\left(-1+x\right)^5}{5}-  \frac{\left(-1+x\right)^6}{6}+\frac{\left(-1+x\right)^7}{7}-\frac{  \left(-1+x\right)^8}{8}+\frac{\left(-1+x\right)^9}{9}-\frac{\left(-1  +x\right)^{10}}{10}+x\]---

Kita juga dapat menggunakan fungsi ``sequence()'' untuk barisan yang kompleks. Fungsi ini menghitung nilai-nilai x{[}n{]} dari semua nilai sebelumnya, x{[}1{]},\ldots,x{[}n-1{]} yang diketahui.

Berikut adalah contoh barisan Fibonacci.

\[x_n=x_{n-1}+x_{n-2}, \quad x_1=2, \quad x_2 =3\]\textgreater sequence(``x{[}n-1{]}+x{[}n-2{]}'',{[}2,3{]},10)

\begin{verbatim}
[2,  3,  5,  8,  13,  21,  34,  55,  89,  144]
\end{verbatim}

Kemudian kita dapat menggunakan fungsi makelist() untuk menghasilkan barisan ekspresi simbolik dengan maxima.

\textgreater\&powerdisp:true

\begin{verbatim}
                                 true
\end{verbatim}

perintah tersebut untuk menampilkan deret pangkat mulai dari suku berpangkat terkecil

Permukaan dalam R\^{}3 terdapat dua macam yakni permukaan linear dan

kuadratik. Setiap permukaan linear berupa bidang datar, sedangkan

permukaan kuadratik berupa bidang lengkung yang kelengkungannya

bergantung atas bentuk persamaannya.

Untuk membuat grafik fungsi tiga dimensi, maka kita dapat menggunakan

perintah ``plot3d()''

\section{Grafik Fungsi Persamaan Linear Dalam Dimensi Tiga}\label{grafik-fungsi-persamaan-linear-dalam-dimensi-tiga}

Bentuk umum persamaan permukaan linear adalah

\[Ax+By+Cz+D=0\]

Contoh grafik persamaan linear di ruang tiga dimensi:

\textgreater deret \&= makelist(taylor(exp(x),x,0,k),k,2,4); \$deret

\[\left[ 1+x+\frac{x^2}{2} , 1+x+\frac{x^2}{2}+\frac{x^3}{6} , 1+x+  \frac{x^2}{2}+\frac{x^3}{6}+\frac{x^4}{24} \right] \]\# FUNGSI MULTIVARIABEL Fungsi multivariabel adalah pemetaan matematis

yang menghubungkan beberapa variabel independen dengan satu variabel dependen. Fungsi ini umumnya dinotasikan sebagai

\[f(x,y) \text{ atau } f(x_1,x_2,...,x_n)\]\[\text{ dimana } x,y \text{ atau } x_1,x_2,...,x_n \text{ adalah variabel independen}\]\[\text{dan f adalah variabel dependen.}\]

Fungsi multivariabel dapat diwakili dalam bentuk peta atau grafik tiga

dimensi.

Contoh fungsi multivariabel :

\[z = x^2 + y^2\]

Dimana variabel bebasnya yaitu x dan y.

Grafik Fungsi Multivariabel Pada fungsi multivariabel, grafik fungsinya merupakan grafik tiga dimensi. Ruang dimensi tiga dilambangkan dengan

\[R^3\]

Permukaan dalam R\^{}3 terdapat dua macam yakni permukaan linear dan

kuadratik. Setiap permukaan linear berupa bidang datar, sedangkan

permukaan kuadratik berupa bidang lengkung yang kelengkungannya

bergantung atas bentuk persamaannya.

Untuk membuat grafik fungsi tiga dimensi, maka kita dapat menggunakan

perintah ``plot3d()''

\section{Grafik Fungsi Persamaan Linear Dalam Dimensi Tiga}\label{grafik-fungsi-persamaan-linear-dalam-dimensi-tiga-1}

Bentuk umum persamaan permukaan linear adalah

\[Ax+By+Cz+D=0\]

Contoh grafik persamaan linear di ruang tiga dimensi:

\textgreater plot3d(``x-2''):

\begin{figure}
\centering
\pandocbounded{\includegraphics[keepaspectratio]{images/KALKULUS_Diva Nagita(23030630024)-391.png}}
\caption{images/KALKULUS\_Diva\%20Nagita(23030630024)-391.png}
\end{figure}

\textgreater plot3d(``2*x+6*y-18''):

\begin{figure}
\centering
\pandocbounded{\includegraphics[keepaspectratio]{images/KALKULUS_Diva Nagita(23030630024)-392.png}}
\caption{images/KALKULUS\_Diva\%20Nagita(23030630024)-392.png}
\end{figure}

\textgreater plot3d(``5*x-2''):

\begin{figure}
\centering
\pandocbounded{\includegraphics[keepaspectratio]{images/KALKULUS_Diva Nagita(23030630024)-393.png}}
\caption{images/KALKULUS\_Diva\%20Nagita(23030630024)-393.png}
\end{figure}

\textgreater plot3d(``2*x+8*y+4*z-18'',implicit=3,r=5):

\begin{figure}
\centering
\pandocbounded{\includegraphics[keepaspectratio]{images/KALKULUS_Diva Nagita(23030630024)-394.png}}
\caption{images/KALKULUS\_Diva\%20Nagita(23030630024)-394.png}
\end{figure}

\section{Grafik Fungsi Kuadratik di Ruang Dimensi Tiga}\label{grafik-fungsi-kuadratik-di-ruang-dimensi-tiga}

Persamaan kuadratik mempunyai rumus umum :

\[Ax^2+By^2+Cz^2+Dxy+Exz+Fyz+Gx+Hy+Iz+J=0\]

Permukaan-permukaan kuadratik dapat berupa permukaan bola, ellipsoida,

paraboloida, tabung ellips, tabung lingkaran, atau tabung parabola.

Contoh grafik fungsi kuadratik di ruang dimensi tiga:

\textgreater plot3d(``2*x\textsuperscript{2+4*y}2''):

\begin{figure}
\centering
\pandocbounded{\includegraphics[keepaspectratio]{images/KALKULUS_Diva Nagita(23030630024)-396.png}}
\caption{images/KALKULUS\_Diva\%20Nagita(23030630024)-396.png}
\end{figure}

\textgreater function f(x,y) := exp(-(2*x\textsuperscript{2+4*y}2))

\textgreater plot3d(``f'', r=5):

\begin{figure}
\centering
\pandocbounded{\includegraphics[keepaspectratio]{images/KALKULUS_Diva Nagita(23030630024)-397.png}}
\caption{images/KALKULUS\_Diva\%20Nagita(23030630024)-397.png}
\end{figure}

\textgreater plot3d(``9*x\textsuperscript{2+4*y}2+9*z\^{}2-36'',implicit=2,r=3):

\begin{figure}
\centering
\pandocbounded{\includegraphics[keepaspectratio]{images/KALKULUS_Diva Nagita(23030630024)-398.png}}
\caption{images/KALKULUS\_Diva\%20Nagita(23030630024)-398.png}
\end{figure}

Gambar diatas merupakan ellipsoid yang persamaannya biasa dinyatakan

dalam

\[\frac{x^2}{a^2}+\frac{y^2}{b^2}+\frac{z^2}{c^2}=1\]

Sedangkan grafik diatas merupakan ellipsoid dengan persamaan :

\[\frac{x^2}{4}+\frac{y^2}{9}+\frac{z^2}{4}=1\]

Namun, untuk memudahkan dalam memplotnya kita hilangkan bentuk

pecahannya dengan mengurangkan kedua ruas dengan satu lalu

mengalikannya dengan 36 sehingga diperoleh

\[9x^2+4y^2+9z^2-36=0\]\textgreater plot3d(``y\textsuperscript{2-x}2'',r=2):

\begin{figure}
\centering
\pandocbounded{\includegraphics[keepaspectratio]{images/KALKULUS_Diva Nagita(23030630024)-402.png}}
\caption{images/KALKULUS\_Diva\%20Nagita(23030630024)-402.png}
\end{figure}

Gambar diatas merupakan paraboloida hiperbolik

\textgreater plot3d(``x\^{}2/y''):

\begin{figure}
\centering
\pandocbounded{\includegraphics[keepaspectratio]{images/KALKULUS_Diva Nagita(23030630024)-403.png}}
\caption{images/KALKULUS\_Diva\%20Nagita(23030630024)-403.png}
\end{figure}

\section{Menggambar kurva perpotongan dari dua persamaan}\label{menggambar-kurva-perpotongan-dari-dua-persamaan}

Di EMT kita juga dapat menggabungkan dua kurva pada satu bidang untuk

menggambarkan perpotongan. Untuk masalah ini kita gunakan fungsi

\[\textgreater{}add\]

dalam prosesnya.

Contoh :

\[x^2+y^2+z-4=0\]\[\text{dengan}\]\[x^2+y^2=1\]\textgreater plot3d(``x\textsuperscript{2+y}2+z-4'',r=5, implicit=3):

\begin{figure}
\centering
\pandocbounded{\includegraphics[keepaspectratio]{images/KALKULUS_Diva Nagita(23030630024)-408.png}}
\caption{images/KALKULUS\_Diva\%20Nagita(23030630024)-408.png}
\end{figure}

\textgreater plot3d(``x\textsuperscript{2+y}2-1'',implicit=3, r=5, \textgreater add):

\begin{figure}
\centering
\pandocbounded{\includegraphics[keepaspectratio]{images/KALKULUS_Diva Nagita(23030630024)-409.png}}
\caption{images/KALKULUS\_Diva\%20Nagita(23030630024)-409.png}
\end{figure}

Dari persamaan diatas, didapatkan perpotongan antara bidang datar dan

bidang lengkung.

\begin{enumerate}
\def\labelenumi{\arabic{enumi}.}
\setcounter{enumi}{2}
\tightlist
\item
\end{enumerate}

\[x^2+y^2=4\]\[\text{ dengan }\]\[y^2+z^2=4\]\textgreater plot3d(``x\textsuperscript{2+y}2-4'',r=5,implicit=3); plot3d(``y\textsuperscript{2+z}2-4'',r=5,implicit=3,\textgreater add):

\begin{figure}
\centering
\pandocbounded{\includegraphics[keepaspectratio]{images/KALKULUS_Diva Nagita(23030630024)-413.png}}
\caption{images/KALKULUS\_Diva\%20Nagita(23030630024)-413.png}
\end{figure}

Didapatkan perpotongan antara dua tabung lingkaran.

\chapter{Turunan Fungsi Multivariabel ** Turunan Fungsi Dua Variabel Turunan}\label{turunan-fungsi-multivariabel-turunan-fungsi-dua-variabel-turunan}

parsial, yaitu turunan fungsi terhadap satu variabel bebas

sementara variabel bebas lainnya dianggap tetap atau konstan.

\begin{itemize}
\tightlist
\item
  Turunan Parsial terhadap f terhadap x di (x0,y0)
\end{itemize}

\[f_x(x_0,y_0)= \lim_{\Delta x \to 0} \frac{f(x_0+\Delta x, y_0)-f(x_0,y_0)}{\Delta x}\]* Turunan Parsial f terhadap y di (x0,y0) * f\_y(x\_0,y\_0)= \lim\_\{\Delta y \to 0\} \frac{f(x_0, y_0+\Delta y)-f(x_0,y_0)}{\Delta y}

\[f_y(x_0,y_0)= \lim_{\Delta y \to 0} \frac{f(x_0, y_0+\Delta y)-f(x_0,y_0)}{\Delta y}\]Contoh soal untuk turunan parsial :

\begin{enumerate}
\def\labelenumi{\arabic{enumi}.}
\tightlist
\item
\end{enumerate}

\[f(x,y)=2xy-(1-x^2)\]\textgreater z \&= 2*x*y-(1-x\^{}2)

\begin{verbatim}
                                  2
                           - 1 + x  + 2 x y
\end{verbatim}

\textgreater\$showev('limit(((2*(x+h)*y)-(1-(x+h)\^{}2))/h,h,0))

\[\lim_{h\rightarrow 0}{\frac{-1+\left(h+x\right)^2+2\,\left(h+x  \right)\,y}{h}}=\lim_{h\rightarrow 0}{\frac{-1+\left(h+x\right)^2+2  \,\left(h+x\right)\,y}{h}}\]Perhitungan akan dilakukan menggunakan diff

\textgreater\&diff(z,x) // z akan diturunkan terhadap x

\begin{verbatim}
                              2 x + 2 y
\end{verbatim}

Karena pada fungsi z terdapat 2xy yang merupakan perkalian, untuk

menghitung turunannya kita gunakan u'v+uv', sehingga turunan dari 2xy:

\[2.1.y+2.x.0=2y\]

Kemudian turunan dari variabel berpangkat yaitu:

u\textsuperscript{n=n.u}\{n-1\}.u'

Jadi turunan dari x\^{}2:

\[x^2=2.x^{2-1}.1=2x\]

Turunan dari konstanta adalah 0

sehingga, turunan dari fungsi

\[f(x,y)=2xy+x^2-1\]

terhadap x adalah

\[f_x(x,y)=2y+2x\]\textgreater\&diff(z,y) // z akan diturunkan terhadap y

\begin{verbatim}
                                 2 x
\end{verbatim}

Seperti pada turunan terhadap x, kita gunakan langkah yang sama namun

kita anggap x konstan.

\[f(x,y)=2xy+x^2-1\]\[f_y(x,y)=2.x.1+0-0\]\[f_y(x,y)=2x\]

Jadi, turunan terhadap y dari f(x,y) adalah 2x.

Sehingga, turunan parsial dari

\[f(x,y)=2xy+x^2-1 \text{ adalah }\]\[f_x(x,y)=2x+2y\]\[f_y(x,y)=2x\] 2. Hitunglah turunan terhadap x dan terhadap y dari

\[f(x,y)=x^2 cos(xy)\]\textgreater expr\&=x\^{}2*cos(x*y); // definisikan fungsi f(x,y)

\textgreater\&diff(expr,x)

\begin{verbatim}
                                     2
                     2 x cos(x y) - x  y sin(x y)
\end{verbatim}

x disini untuk menandakan bahwa fungsi diturunkan terhadap x dan y

dianggap konstan.

\textgreater\&diff(expr,y)

\begin{verbatim}
                               3
                            - x  sin(x y)
\end{verbatim}

y menandakan bahwa fungsi diturunkan terhadap y dan x dianggap\\
konstan.

Jadi, turunan dari fungsi

\[x^2 cos(xy)\]\[\text{terhadap x  : } 2x.cos(xy)-x^2y.sin(xy)\]\[\text{terhadap y  : } -x^3sin(xy)\]\#\# Turunan Orde Tinggi

Turunan parsial kedua meliputi :

\[f_{xx}=\frac{\partial^2 f}{\partial x^2}\]

Fungsi f diturunkan terhadap x kemudian turunan pertama diturunkan

lagi terhadap x.

\[f_{xy}=\frac{\partial^2 f}{\partial y  \partial x}\]

Fungsi f diturunkan terhadap x kemudian turunan pertamanya diturunkan

lagi terhadap y.

\[f_{yy}=\frac{\partial^2 f}{\partial y^2}\]

Fungsi f diturunkan terhadap y kemudian turunan pertama diturunkan

lagi terhadap y.

\[f_{yx}=\frac{\partial^2 f}{\partial x  \partial y}\]

Fungsi f diturunkan terhadap y kemudian turunan pertama diturunkan

lagi terhadap x.

Turunan parsial tingkat tiga dan lebih tinggi didefinisikan dengan

cara yang sama dan cara penulisannya pun serupa. Jadi, jika suatu

fungsi dua variabel x dan y, turunan parsial ketiga f yang diperoleh

dengan mendiferensialkan f secara parsial, pertamakali terhadap x dan

kemudian terhadap y, akan ditunjukkan oleh

\[f_{xyy}=\frac{\partial^3 f}{\partial y^2 \partial x}\]

Seluruhnya, terdapat delapan turunan parsial ketiga.

Contoh :

Carilah

\[f_{xx} \text{dan} f_{yy}\]

dari fungsi berikut

\[f(x,y)=xe^y-sin(\frac{x}{y})+x^3y^2\]\# Latihan Soal

\textgreater function a(x,y) \ldots{}

\begin{verbatim}
return x^2+y^2-24
endfunction
\end{verbatim}

\textgreater function q(x,y) \ldots{}

\begin{verbatim}
return y^2/(x^2/3)
endfunction
\end{verbatim}

\textgreater q(4,2), q(2,3), q(4,3)

\begin{verbatim}
0.75
6.75
1.6875
\end{verbatim}

\textgreater a(2,1), a(5,4), a(2,4)

\begin{verbatim}
-19
17
-4
\end{verbatim}

\backmatter
\end{document}
